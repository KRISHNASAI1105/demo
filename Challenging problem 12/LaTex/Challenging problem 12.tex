\documentclass[journal,12pt,twocolumn]{IEEEtran}
\usepackage{setspace}
\usepackage{gensymb}
\singlespacing
\usepackage[cmex10]{amsmath}

\usepackage{relsize}
\usepackage{amsthm}
\usepackage{hyperref}
\hypersetup{
    colorlinks=true,
    linkcolor=blue,
    filecolor=magenta,      
    urlcolor=cyan,
}
\urlstyle{same}
\usepackage{placeins}
\usepackage{newtxtext}
\usepackage{mathrsfs}
\usepackage{txfonts}
\usepackage{stfloats}
\usepackage{bm}
\usepackage{cite}
\usepackage{cases}
\usepackage{subfig}
\usepackage{longtable}
\usepackage{multirow}
\usepackage{enumitem}
\usepackage{mathtools}
\usepackage{steinmetz}
\usepackage{tikz}
\usepackage{circuitikz}
\usepackage{verbatim}
\usepackage{tfrupee}
\usepackage[breaklinks=true]{hyperref}
\usepackage{booktabs}
\usepackage{graphicx}
\usepackage{tkz-euclide}
\usetikzlibrary{shapes,backgrounds}
\usepackage{verbatim}
\usetikzlibrary{calc,math}
\usepackage{listings}
    \usepackage{color}                                            %%
    \usepackage{array}                                            %%
    \usepackage{longtable}                                        %%
    \usepackage{calc}                                             %%
    \usepackage{multirow}                                         %%
    \usepackage{hhline}                                           %%
    \usepackage{ifthen}                                           %%
    \usepackage{lscape}     
\usepackage{multicol}
\usepackage{chngcntr}
\usepackage{mdframed}
\DeclareMathOperator*{\Res}{Res}

\renewcommand\thesection{\arabic{section}}
\renewcommand\thesubsection{\thesection.\arabic{subsection}}
\renewcommand\thesubsubsection{\thesubsection.\arabic{subsubsection}}

\renewcommand\thesectiondis{\arabic{section}}
\renewcommand\thesubsectiondis{\thesectiondis.\arabic{subsection}}
\renewcommand\thesubsubsectiondis{\thesubsectiondis.\arabic{subsubsection}}


\hyphenation{op-tical net-works semi-conduc-tor}
\def\inputGnumericTable{}                                 %%

\lstset{
%language=C,
frame=single, 
breaklines=true,
columns=fullflexible
}
\usepackage{chngcntr}
\counterwithin{figure}{section}

\title{AI1103 \\ Assignment 4}
\author{Nagubandi Krishna Sai \\ MS20BTECH11014}

\begin{document}
\newtheorem{theorem}{Theorem}[section]
\newtheorem{problem}{Problem}
\newtheorem{proposition}{Proposition}[section]
\newtheorem{lemma}{Lemma}[section]
\newtheorem{corollary}[theorem]{Corollary}
\newtheorem{example}{Example}[section]
\newtheorem{definition}[problem]{Definition}

\newcommand{\BEQA}{\begin{eqnarray}}
\newcommand{\EEQA}{\end{eqnarray}}
\newcommand{\define}{\stackrel{\triangle}{=}}
\bibliographystyle{IEEEtran}
\raggedbottom
\setlength{\parindent}{0pt}
\providecommand{\mbf}{\mathbf}
\providecommand{\pr}[1]{\ensuremath{\Pr\left(#1\right)}}
\providecommand{\qfunc}[1]{\ensuremath{Q\left(#1\right)}}
\providecommand{\sbrak}[1]{\ensuremath{{}\left[#1\right]}}
\providecommand{\lsbrak}[1]{\ensuremath{{}\left[#1\right.}}
\providecommand{\rsbrak}[1]{\ensuremath{{}\left.#1\right]}}
\providecommand{\brak}[1]{\ensuremath{\left(#1\right)}}
\providecommand{\lbrak}[1]{\ensuremath{\left(#1\right.}}
\providecommand{\rbrak}[1]{\ensuremath{\left.#1\right)}}
\providecommand{\cbrak}[1]{\ensuremath{\left\{#1\right\}}}
\providecommand{\lcbrak}[1]{\ensuremath{\left\{#1\right.}}
\providecommand{\rcbrak}[1]{\ensuremath{\left.#1\right\}}}
\theoremstyle{remark}
\newtheorem{rem}{Remark}
\newcommand{\sgn}{\mathop{\mathrm{sgn}}}
\newcommand{\comb}[2]{{}^{#1}\mathrm{C}_{#2}}
\providecommand{\abs}[1]{\vert#1\vert}
\providecommand{\res}[1]{\Res\displaylimits_{#1}} 
\providecommand{\norm}[1]{\lVert#1\rVert}
%\providecommand{\norm}[1]{\lVert#1\rVert}
\providecommand{\mtx}[1]{\mathbf{#1}}
\providecommand{\mean}[1]{E[ #1 ]}
\providecommand{\fourier}{\overset{\mathcal{F}}{ \rightleftharpoons}}
%\providecommand{\hilbert}{\overset{\mathcal{H}}{ \rightleftharpoons}}
\providecommand{\system}{\overset{\mathcal{H}}{ \longleftrightarrow}}
	%\newcommand{\solution}[2]{\textbf{Solution:}{#1}}
\newcommand{\solution}{\noindent \textbf{Solution: }}
\newcommand{\cosec}{\,\text{cosec}\,}
\providecommand{\dec}[2]{\ensuremath{\overset{#1}{\underset{#2}{\gtrless}}}}
\newcommand{\myvec}[1]{\ensuremath{\begin{pmatrix}#1\end{pmatrix}}}
\newcommand{\mydet}[1]{\ensuremath{\begin{vmatrix}#1\end{vmatrix}}}
\numberwithin{equation}{subsection}
\makeatletter
\@addtoreset{figure}{problem}
\makeatother
\let\StandardTheFigure\thefigure
\let\vec\mathbf
\renewcommand{\thefigure}{\theproblem}
\def\putbox#1#2#3{\makebox[0in][l]{\makebox[#1][l]{}\raisebox{\baselineskip}[0in][0in]{\raisebox{#2}[0in][0in]{#3}}}}
     \def\rightbox#1{\makebox[0in][r]{#1}}
     \def\centbox#1{\makebox[0in]{#1}}
     \def\topbox#1{\raisebox{-\baselineskip}[0in][0in]{#1}}
     \def\midbox#1{\raisebox{-0.5\baselineskip}[0in][0in]{#1}}
\vspace{3cm}
\title{\textbf{AI1103} \\ \textbf{\text{Challenging Problem 12}}}
\author{Nagubandi Krishna Sai \\ MS20BTECH11014}
\maketitle
\newpage
\bigskip
\renewcommand{\thefigure}{\theenumi}
\renewcommand{\thetable}{\theenumi}


\textbf{Download LaTex file from below link :} 

\begin{lstlisting}
https://github.com/KRISHNASAI1105/demo/blob/main/Challenging%20problem%2012/LaTex/Challenging%20problem%2012.tex
\end{lstlisting}

\subsection*{\boldsymbol{UGC\ NET\ JUNE\ 2019\ Q.51}} 
\begin{flushleft} Consider a Markov chain with state space \{0,1,2,3,4\} and transition matrix \\
\begin{math}
P=\left[
\begin{array}{c c c c c}
    1 & 0 & 0 & 0 & 0\\
    \frac{1}{3} & \frac{1}{3}  & \frac{1}{3}  & 0 & 0 \\
    0 & \frac{1}{3}  & \frac{1}{3}  & \frac{1}{3} & 0 \\
    0 & 0 & \frac{1}{3} & \frac{1}{3}  & \frac{1}{3}  \\
    0 & 0 & 0 & 0 & 1
\end{array}\right]
\end{math} \\
Then $lim_{n\to\infty} p_{23}^{(n)}$ equals
\begin{enumerate}
    \item \dfrac{1}{3} \\
    \item \dfrac{1}{2} \\
    \item 0 \\
    \item 1
\end{enumerate}
\end{flushleft} \vspace{-5mm}

\subsection*{\boldsymbol{Solution}} 
\begin{flushleft} 
\begin{math}
P = \left[
\begin{array}{c c c c c}
    1 & 0 & 0 & 0 & 0 \\
    \frac{1}{3} & \frac{1}{3}  & \frac{1}{3}  & 0 & 0 \\
    0 & \frac{1}{3}  & \frac{1}{3}  & \frac{1}{3} & 0 \\
    0 & 0 & \frac{1}{3} & \frac{1}{3}  & \frac{1}{3}  \\
    0 & 0 & 0 & 0 & 1
\end{array}\right]
\end{math} \\
For eigenvalue calculation, \\
\begin{center}
    $|A-\lambda I|$ = 0 & $A=P^\intercal$
\end{center} \\
\begin{math}
A = \left[
\begin{array}{c c c c c}
    1 & \frac{1}{3} & 0 & 0 & 0 \\
    0 & \frac{1}{3}  & \frac{1}{3}  & 0 & 0 \\
    0 & \frac{1}{3}  & \frac{1}{3}  & \frac{1}{3} & 0 \\
    0 & 0 & \frac{1}{3} & \frac{1}{3}  & 0  \\
    0 & 0 & 0 & \frac{1}{3} & 1
\end{array}\right]
\end{math} \\
\begin{math}
A - \lambda I = \left[
\begin{array}{c c c c c}
    1-\lambda & \frac{1}{3} & 0 & 0 & 0 \\
    0 & \frac{1}{3}-\lambda  & \frac{1}{3}  & 0 & 0 \\
    0 & \frac{1}{3}  & \frac{1}{3}-\lambda  & \frac{1}{3} & 0 \\
    0 & 0 & \frac{1}{3} & \frac{1}{3}-\lambda  & 0  \\
    0 & 0 & 0 & \frac{1}{3} & 1-\lambda
\end{array}\right]
\end{math} \\
\begin{align}
|A-\lambda I| &= \nonumber\brak{1-\lambda}\left|
\begin{array}{c c c c}
     \frac{1}{3}-\lambda  & \frac{1}{3}  & 0 & 0 \\
     \frac{1}{3}  & \frac{1}{3}-\lambda  & \frac{1}{3} & 0 \\
     0 & \frac{1}{3} & \frac{1}{3}-\lambda  & 0  \\
     0 & 0 & \frac{1}{3} & 1-\lambda
\end{array}\right| \\ 
& - \dfrac{1}{3}\left|
\begin{array}{c c c c}
     0 & \frac{1}{3}  & 0 & 0 \\
     0 & \frac{1}{3}-\lambda  & \frac{1}{3} & 0 \\
     0 & \frac{1}{3} & \frac{1}{3}-\lambda  & 0  \\
     0 & 0 & \frac{1}{3} & 1-\lambda
\end{array}\right| \\
&= \nonumber\brak{1-\lambda} \brak{\dfrac{1}{3}-\lambda}\left|
\begin{array}{c c c}
     \frac{1}{3}-\lambda & \frac{1}{3} & 0 \\
     \frac{1}{3} & \frac{1}{3}-\lambda & 0  \\
     0 & \frac{1}{3} & 1-\lambda
\end{array}\right| \\
& \nonumber - \dfrac{1}{3}\brak{1-\lambda}\left|
\begin{array}{c c c}
     \frac{1}{3} & \frac{1}{3} & 0 \\
     0 & \frac{1}{3}-\lambda & 0  \\
     0 & \frac{1}{3} & 1-\lambda
\end{array}\right| \\
& +\dfrac{1}{9}\left|
\begin{array}{c c c}
     \frac{1}{3}-\lambda  & \frac{1}{3} & 0 \\
     \frac{1}{3} & \frac{1}{3}-\lambda  & 0  \\
     0 & \frac{1}{3} & 1-\lambda
\end{array}\right| \\
&= \nonumber\brak{1-\lambda}^2\brak{\dfrac{1}{3}-\lambda}^3 - \dfrac{2}{9} \brak{1-\lambda}^2\brak{\dfrac{1}{3}-\lambda} \\
& + \dfrac{1}{9} \brak{1-\lambda}\brak{\dfrac{1}{3}-\lambda}^2 - \dfrac{1}{81} \brak{1-\lambda} \\
&= \nonumber\brak{1-\lambda}^2\brak{\dfrac{1}{3}-\lambda}\sbrak{\brak{\dfrac{1}{3}-\lambda}^2 - \dfrac{2}{9}} \\
& + \dfrac{1}{9}\brak{1-\lambda}\sbrak{\brak{\dfrac{1}{3}-\lambda}^2-\dfrac{1}{9}} \\
&= \dfrac{-27{\lambda}^5+81{\lambda}^4-84{\lambda}^3+32{\lambda}^2-{\lambda}-1}{27} \\
&= \brak{1-\lambda}^2\brak{\dfrac{1}{3}-\lambda}\brak{{\lambda}^2-\dfrac{2\lambda}{3}-\dfrac{1}{9}}
\end{align}
For eigenvalue calculation, \\
\begin{center}
    $|A-\lambda I|$ = 0 
\end{center} \\
Eigenvalues are,\\
\begin{enumerate}
    \item $\lambda_1$ = 1 \\
    \item $\lambda_2$ = 1 \\
    \item $\lambda_3$ = \dfrac{1}{3} \\
    \item $\lambda_4$ = \dfrac{1-\sqrt{2}}{3} \\
    \item $\lambda_5$ = \dfrac{1+\sqrt{2}}{3} \\
\end{enumerate}
Corresponding Eigenvectors are: \\
\begin{enumerate}
\item Eigenvector $V_1$,$V_2$ for $\lambda_1$=1(multiplicity =2) :\\
\begin{math}
\text{So, for } \lambda=1 \text{ we get 2 eigenvectors.} \\
(A-\lambda_1).V_1=0 \\
\left[
\begin{array}{c c c c c}
    1-\lambda_1 & \frac{1}{3} & 0 & 0 & 0 \\
    0 & \frac{1}{3}-\lambda_1  & \frac{1}{3}  & 0 & 0 \\
    0 & \frac{1}{3}  & \frac{1}{3}-\lambda_1  & \frac{1}{3} & 0 \\
    0 & 0 & \frac{1}{3} & \frac{1}{3}-\lambda_1  & 0  \\
    0 & 0 & 0 & \frac{1}{3} & 1-\lambda_1
\end{array}\right].V_1=0 \\
\left[
\begin{array}{c c c c c}
    0 & \frac{1}{3} & 0 & 0 & 0 \\
    0 & \frac{-2}{3}  & \frac{1}{3}  & 0 & 0 \\
    0 & \frac{1}{3}  & \frac{-2}{3} & \frac{1}{3} & 0 \\
    0 & 0 & \frac{1}{3} & \frac{-2}{3}  & 0  \\
    0 & 0 & 0 & \frac{1}{3} & 0
\end{array}\right].\left[
\begin{array}{c}
     a_1\\
     b_1\\
     c_1\\
     d_1\\
     e_1
\end{array}\right]=0 \\
Null space=a_1\left[
\begin{array}{c}
      1\\
      0\\
      0\\
      0\\
      0
\end{array}\right]+e_1\left[
\begin{array}{c}
      0\\
      0\\
      0\\
      0\\
      1
\end{array}\right] \\
\text{So, eigenvectors are,} \\
V_1=\left[
\begin{array}{c}
      1\\
      0\\
      0\\
      0\\
      0
\end{array}\right],V_2=\left[
\begin{array}{c}
      0\\
      0\\
      0\\
      0\\
      1
\end{array}\right]
\end{math} \\
\item Eigenvector $V_3$ for $\lambda_3$=\dfrac{1}{3}(multiplicity =1) :\\
\begin{math}
(A-\lambda_3).V_3=0 \\
\left[
\begin{array}{c c c c c}
    1-\lambda_3 & \frac{1}{3} & 0 & 0 & 0 \\
    0 & \frac{1}{3}-\lambda_3  & \frac{1}{3}  & 0 & 0 \\
    0 & \frac{1}{3}  & \frac{1}{3}-\lambda_3  & \frac{1}{3} & 0 \\
    0 & 0 & \frac{1}{3} & \frac{1}{3}-\lambda_3  & 0  \\
    0 & 0 & 0 & \frac{1}{3} & 1-\lambda_3
\end{array}\right].V_3=0 \\
\left[
\begin{array}{c c c c c}
    \frac{2}{3} & \frac{1}{3} & 0 & 0 & 0 \\
    0 & 0 & \frac{1}{3} & 0 & 0 \\
    0 & \frac{1}{3} & 0 & \frac{1}{3} & 0 \\
    0 & 0 & \frac{1}{3} & 0 & 0  \\
    0 & 0 & 0 & \frac{1}{3} & \frac{2}{3}
\end{array}\right].\left[
\begin{array}{c}
     a_3\\
     b_3\\
     c_3\\
     d_3\\
     e_3
\end{array}\right]=0 \\
V_3=\left[
\begin{array}{c}
     a_3\\
     b_3\\
     c_3\\
     d_3\\
     e_3
\end{array}\right]=
\left[
\begin{array}{c}
      -1\\
      2\\
      0\\
      -2\\
      1
\end{array}\right]
\end{math} \\
\item Eigenvector $V_4$ for $\lambda_4$=\dfrac{1-\sqrt{2}}{3}(multiplicity =1) :\\
\begin{math}
(A-\lambda_4).V_4=0 \\
\left[
\begin{array}{c c c c c}
    1-\lambda_4 & \frac{1}{3} & 0 & 0 & 0 \\
    0 & \frac{1}{3}-\lambda_4 & \frac{1}{3}  & 0 & 0 \\
    0 & \frac{1}{3}  & \frac{1}{3}-\lambda_4 & \frac{1}{3} & 0 \\
    0 & 0 & \frac{1}{3} & \frac{1}{3}-\lambda_4 & 0  \\
    0 & 0 & 0 & \frac{1}{3} & 1-\lambda_4
\end{array}\right].V_4=0 \\
\left[
\begin{array}{c c c c c}
    \dfrac{2+\sqrt{2}}{3} & \frac{1}{3} & 0 & 0 & 0 \\
    0 & \frac{\sqrt{2}}{3} & \frac{1}{3} & 0 & 0 \\
    0 & \frac{1}{3} & \frac{\sqrt{2}}{3} & \frac{1}{3} & 0 \\
    0 & 0 & \frac{1}{3} & \frac{\sqrt{2}}{3} & 0  \\
    0 & 0 & 0 & \frac{1}{3} & \dfrac{2+\sqrt{2}}{3}
\end{array}\right].\left[
\begin{array}{c}
     a_4\\
     b_4\\
     c_4\\
     d_4\\
     e_4
\end{array}\right]=0 \\
V_4=\left[
\begin{array}{c}
     a_4\\
     b_4\\
     c_4\\
     d_4\\
     e_4
\end{array}\right]=
\left[
\begin{array}{c}
      1\\
      -2-\sqrt{2}\\
      2+2\sqrt{2}\\
      -2-\sqrt{2}\\
      1
\end{array}\right]
\end{math} \\
\item Eigenvector $V_5$ for $\lambda_5$=\dfrac{1+\sqrt{2}}{3}(multiplicity =1) :\\
\begin{math}
(A-\lambda_5).V_5=0 \\
\left[
\begin{array}{c c c c c}
    1-\lambda_5 & \frac{1}{3} & 0 & 0 & 0 \\
    0 & \frac{1}{3}-\lambda_5 & \frac{1}{3}  & 0 & 0 \\
    0 & \frac{1}{3}  & \frac{1}{3}-\lambda_5 & \frac{1}{3} & 0 \\
    0 & 0 & \frac{1}{3} & \frac{1}{3}-\lambda_5 & 0  \\
    0 & 0 & 0 & \frac{1}{3} & 1-\lambda_5
\end{array}\right].V_5=0 \\
\left[
\begin{array}{c c c c c}
    \dfrac{2-\sqrt{2}}{3} & \frac{1}{3} & 0 & 0 & 0 \\
    0 & \frac{-\sqrt{2}}{3} & \frac{1}{3} & 0 & 0 \\
    0 & \frac{1}{3} & \frac{-\sqrt{2}}{3} & \frac{1}{3} & 0 \\
    0 & 0 & \frac{1}{3} & \frac{-\sqrt{2}}{3} & 0  \\
    0 & 0 & 0 & \frac{1}{3} & \dfrac{2-\sqrt{2}}{3}
\end{array}\right].\left[
\begin{array}{c}
     a_5\\
     b_5\\
     c_5\\
     d_5\\
     e_5
\end{array}\right]=0 \\
V_5=\left[
\begin{array}{c}
     a_5\\
     b_5\\
     c_5\\
     d_5\\
     e_5
\end{array}\right]=
\left[
\begin{array}{c}
      1\\
      -2+\sqrt{2}\\
      2-2\sqrt{2}\\
      -2+\sqrt{2}\\
      1
\end{array}\right]
\end{math} \\
\end{enumerate}
\begin{center}
\begin{math}
    V = \left[ V_1 V_2 V_3 V_4 V_5\right]
\end{math}
\end{center}
\begin{math}
V=\left[
\begin{array}{c c c c c}
    1 & 0 & -1 & 1 & 1\\
    0 & 0 & 2 & -2-\sqrt{2} & -2+\sqrt{2} \\
    0 & 0 & 0 & 2+2\sqrt{2} & 2-2\sqrt{2} \\
    0 & 0 & -2 & -2-\sqrt{2} & -2+\sqrt{2}  \\
    0 & 1 & 1 & 1 & 1
\end{array}\right]
\end{math} 
\begin{math}
V^{-1} = \dfrac{1}{8}\left[
\begin{array}{c c c c c}
    8 & 6 & 4 & 2 & 0\\
    0 & 2 & 4 & 6 & 8 \\
    0 & 2 & 0 & -2 & 0 \\
    0 & -2+\sqrt{2} & -2+2\sqrt{2} & -2+\sqrt{2} & 0  \\
    0 & -2-\sqrt{2} & -2-2\sqrt{2} & -2-\sqrt{2} & 0
\end{array}\right]
\end{math}
\begin{math}
V^{-1}AV = \left[
\begin{array}{c c c c c}
    1 & 0 & 0 & 0 & 0\\
    0 & 1 & 0 & 0 & 0 \\
    0 & 0 & \dfrac{1}{3} & 0 & 0 \\
    0 & 0 & 0 & \dfrac{1-\sqrt{2}}{3} & 0 \\
    0 & 0 & 0 & 0 & \dfrac{1+\sqrt{2}}{3}
\end{array}\right]
\end{math} \\
The above matrix is a diagonal matrix, with eigenvalues has diagonal elements.\\
\begin{center}
$V^{-1}AV$\times$V^{-1}AV$\times......(\text{n times})$V^{-1}AV$= $V^{-1}A^{n}V$
\end{center}
\textbf{As the $2^{nd}$ row $3^{rd}$ column element in the above matrix is Zero.} \\
\begin{center}
    $V^{-1}A^{n}V$ = B
\end{center} \\
Let, B be some matrix after the whole multiplication. \\
\begin{center}
    $A^{n}$ = $VBV^{-1}$ 
\end{center} \\
V,$V^{-1}$ are left and right eigenvectors. Even if we don't know B matrix, $A^{n}$ resembles $P^{n}$. \\
As $\lim_{n \to \infty } {Pr_{n}}^n$ approaches a matrix which has structure that all rows of matrix are identical.  \\
\therefore \text{Hence}, \lim_{n\to\infty} p_{23}^{(n)} = 0. 
\begin{center}
  $\therefore$ \boxed{\text{\textbf{Option 3} is \textbf{correct} answer.}}
\end{center}
\end{flushleft}
\end{document}