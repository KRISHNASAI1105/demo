\documentclass[journal,12pt,twocolumn]{IEEEtran}
\usepackage{setspace}
\usepackage{gensymb}
\singlespacing
\usepackage[cmex10]{amsmath}

\usepackage{relsize}
\usepackage{amsthm}
\usepackage{hyperref}
\hypersetup{
    colorlinks=true,
    linkcolor=blue,
    filecolor=magenta,      
    urlcolor=cyan,
}
\urlstyle{same}
\usepackage{placeins}
\usepackage{newtxtext}
\usepackage{mathrsfs}
\usepackage{txfonts}
\usepackage{stfloats}
\usepackage{bm}
\usepackage{cite}
\usepackage{cases}
\usepackage{subfig}
\usepackage{longtable}
\usepackage{multirow}
\usepackage{enumitem}
\usepackage{mathtools}
\usepackage{steinmetz}
\usepackage{tikz}
\usepackage{circuitikz}
\usepackage{verbatim}
\usepackage{tfrupee}
\usepackage[breaklinks=true]{hyperref}
\usepackage{booktabs}
\usepackage{graphicx}
\usepackage{tkz-euclide}
\usetikzlibrary{shapes,backgrounds}
\usepackage{verbatim}
\usetikzlibrary{calc,math}
\usepackage{listings}
    \usepackage{color}                                            %%
    \usepackage{array}                                            %%
    \usepackage{longtable}                                        %%
    \usepackage{calc}                                             %%
    \usepackage{multirow}                                         %%
    \usepackage{hhline}                                           %%
    \usepackage{ifthen}                                           %%
    \usepackage{lscape}     
\usepackage{multicol}
\usepackage{chngcntr}
\usepackage{mdframed}
\DeclareMathOperator*{\Res}{Res}

\renewcommand\thesection{\arabic{section}}
\renewcommand\thesubsection{\thesection.\arabic{subsection}}
\renewcommand\thesubsubsection{\thesubsection.\arabic{subsubsection}}

\renewcommand\thesectiondis{\arabic{section}}
\renewcommand\thesubsectiondis{\thesectiondis.\arabic{subsection}}
\renewcommand\thesubsubsectiondis{\thesubsectiondis.\arabic{subsubsection}}


\hyphenation{op-tical net-works semi-conduc-tor}
\def\inputGnumericTable{}                                 %%

\lstset{
%language=C,
frame=single, 
breaklines=true,
columns=fullflexible
}
\usepackage{chngcntr}
\counterwithin{figure}{section}

\title{AI1103 \\ Assignment 4}
\author{Nagubandi Krishna Sai \\ MS20BTECH11014}

\begin{document}
\newtheorem{theorem}{Theorem}[section]
\newtheorem{problem}{Problem}
\newtheorem{proposition}{Proposition}[section]
\newtheorem{lemma}{Lemma}[section]
\newtheorem{corollary}[theorem]{Corollary}
\newtheorem{example}{Example}[section]
\newtheorem{definition}[problem]{Definition}

\newcommand{\BEQA}{\begin{eqnarray}}
\newcommand{\EEQA}{\end{eqnarray}}
\newcommand{\define}{\stackrel{\triangle}{=}}
\bibliographystyle{IEEEtran}
\raggedbottom
\setlength{\parindent}{0pt}
\providecommand{\mbf}{\mathbf}
\providecommand{\pr}[1]{\ensuremath{\Pr\left(#1\right)}}
\providecommand{\qfunc}[1]{\ensuremath{Q\left(#1\right)}}
\providecommand{\sbrak}[1]{\ensuremath{{}\left[#1\right]}}
\providecommand{\lsbrak}[1]{\ensuremath{{}\left[#1\right.}}
\providecommand{\rsbrak}[1]{\ensuremath{{}\left.#1\right]}}
\providecommand{\brak}[1]{\ensuremath{\left(#1\right)}}
\providecommand{\lbrak}[1]{\ensuremath{\left(#1\right.}}
\providecommand{\rbrak}[1]{\ensuremath{\left.#1\right)}}
\providecommand{\cbrak}[1]{\ensuremath{\left\{#1\right\}}}
\providecommand{\lcbrak}[1]{\ensuremath{\left\{#1\right.}}
\providecommand{\rcbrak}[1]{\ensuremath{\left.#1\right\}}}
\theoremstyle{remark}
\newtheorem{rem}{Remark}
\newcommand{\sgn}{\mathop{\mathrm{sgn}}}
\newcommand{\comb}[2]{{}^{#1}\mathrm{C}_{#2}}
\providecommand{\abs}[1]{\vert#1\vert}
\providecommand{\res}[1]{\Res\displaylimits_{#1}} 
\providecommand{\norm}[1]{\lVert#1\rVert}
%\providecommand{\norm}[1]{\lVert#1\rVert}
\providecommand{\mtx}[1]{\mathbf{#1}}
\providecommand{\mean}[1]{E[ #1 ]}
\providecommand{\fourier}{\overset{\mathcal{F}}{ \rightleftharpoons}}
%\providecommand{\hilbert}{\overset{\mathcal{H}}{ \rightleftharpoons}}
\providecommand{\system}{\overset{\mathcal{H}}{ \longleftrightarrow}}
	%\newcommand{\solution}[2]{\textbf{Solution:}{#1}}
\newcommand{\solution}{\noindent \textbf{Solution: }}
\newcommand{\cosec}{\,\text{cosec}\,}
\providecommand{\dec}[2]{\ensuremath{\overset{#1}{\underset{#2}{\gtrless}}}}
\newcommand{\myvec}[1]{\ensuremath{\begin{pmatrix}#1\end{pmatrix}}}
\newcommand{\mydet}[1]{\ensuremath{\begin{vmatrix}#1\end{vmatrix}}}
\numberwithin{equation}{subsection}
\makeatletter
\@addtoreset{figure}{problem}
\makeatother
\let\StandardTheFigure\thefigure
\let\vec\mathbf
\renewcommand{\thefigure}{\theproblem}
\def\putbox#1#2#3{\makebox[0in][l]{\makebox[#1][l]{}\raisebox{\baselineskip}[0in][0in]{\raisebox{#2}[0in][0in]{#3}}}}
     \def\rightbox#1{\makebox[0in][r]{#1}}
     \def\centbox#1{\makebox[0in]{#1}}
     \def\topbox#1{\raisebox{-\baselineskip}[0in][0in]{#1}}
     \def\midbox#1{\raisebox{-0.5\baselineskip}[0in][0in]{#1}}
\vspace{3cm}
\title{\textbf{AI1103} \\ \textbf{\text{Challenging Problem 12}}}
\author{Nagubandi Krishna Sai \\ MS20BTECH11014}
\maketitle
\newpage
\bigskip
\renewcommand{\thefigure}{\theenumi}
\renewcommand{\thetable}{\theenumi}


\textbf{Download LaTex file from below link :} 

\begin{lstlisting}
https://github.com/KRISHNASAI1105/demo/blob/main/Challenging%20problem%207/LaTex/Challenging%20problem%207.tex
\end{lstlisting}

\subsection*{\boldsymbol{UGC\ NET\ JUNE\ 2019\ Q.51}} 
\begin{flushleft} Consider a Markov chain with state space \{0,1,2,3,4\} and transition matrix \\
\begin{math}
P=\left[
\begin{array}{c c c c c}
    1 & 0 & 0 & 0 & 0\\
    \frac{1}{3} & \frac{1}{3}  & \frac{1}{3}  & 0 & 0 \\
    0 & \frac{1}{3}  & \frac{1}{3}  & \frac{1}{3} & 0 \\
    0 & 0 & \frac{1}{3} & \frac{1}{3}  & \frac{1}{3}  \\
    0 & 0 & 0 & 0 & 1
\end{array}\right]
\end{math} \\
Then $lim_{n\to\infty} p_{23}^{(n)}$ equals
\begin{enumerate}
    \item \dfrac{1}{3} \\
    \item \dfrac{1}{2} \\
    \item 0 \\
    \item 1
\end{enumerate}
\end{flushleft} \vspace{-5mm}

\subsection*{\boldsymbol{Solution}} 
\begin{flushleft} 
\begin{math}
P = \left[
\begin{array}{c c c c c}
    1 & 0 & 0 & 0 & 0 \\
    \frac{1}{3} & \frac{1}{3}  & \frac{1}{3}  & 0 & 0 \\
    0 & \frac{1}{3}  & \frac{1}{3}  & \frac{1}{3} & 0 \\
    0 & 0 & \frac{1}{3} & \frac{1}{3}  & \frac{1}{3}  \\
    0 & 0 & 0 & 0 & 1
\end{array}\right]
\end{math} \\
\begin{math}
P^2 = \left[
\begin{array}{c c c c c}
    1 & 0 & 0 & 0 & 0 \\
    \frac{4}{9} & \frac{2}{9}  & \frac{2}{9}  & \frac{1}{9} & 0 \\
    \frac{1}{9} & \frac{2}{9}  & \frac{1}{3}  & \frac{2}{9} & \frac{1}{9} \\
    0 & \frac{1}{9} & \frac{2}{9} & \frac{2}{9}  & \frac{4}{9}  \\
    0 & 0 & 0 & 0 & 1
\end{array}\right]
\end{math} \\
\begin{math}
P^3 = \left[
\begin{array}{c c c c c}
    1 & 0 & 0 & 0 & 0 \\
    \frac{14}{27} & \frac{4}{27}  & \frac{5}{27}  & \frac{1}{9} & \frac{1}{27}\\
    \frac{5}{27} & \frac{5}{27}  & \frac{7}{27}  & \frac{5}{27} & \frac{5}{27}\\
    \frac{1}{27} & \frac{1}{9} & \frac{5}{27} & \frac{4}{27}  & \frac{14}{27}\\
    0 & 0 & 0 & 0 & 1
\end{array}\right]
\end{math} \\
\begin{math}
P^4 = \left[
\begin{array}{c c c c c}
    1 & 0 & 0 & 0 & 0 \\
    \frac{46}{81} & \frac{1}{9}  & \frac{4}{27}  & \frac{8}{81} & \frac{2}{27}\\
    \frac{20}{81} & \frac{4}{27}  & \frac{17}{81}  & \frac{4}{27} & \frac{20}{81}\\
    \frac{2}{27} & \frac{8}{81} & \frac{4}{27} & \frac{1}{9}  & \frac{46}{81}\\
    0 & 0 & 0 & 0 & 1
\end{array}\right]
\end{math} \\
\begin{math}
P^n = \left[
\begin{array}{c c c c c}
    1 & 0 & 0 & 0 & 0 \\
    ... & ... & \frac{\frac{1}{30}n^5-\frac{1}{3}n^4+\frac{3}{2}n^3-\frac{8}{3}n^2+\frac{37}{15}n}{3^n}  & ... & ...\\
    ... & ...  & \frac{(1+\sqrt{2})^n + (1-\sqrt{2})^n}{3^n}  & ... & ...\\
    ... & ... & \frac{\frac{1}{30}n^5-\frac{1}{3}n^4+\frac{3}{2}n^3-\frac{8}{3}n^2+\frac{37}{15}n}{3^n} & ...  & ...\\
    0 & 0 & 0 & 0 & 1
\end{array}\right]
\end{math} \\
\boxed{\textbf{As we only require $2^{nd}$ row $3^{rd}$ column element in the}}\\
\boxed{\textbf{$p^n$ matrix, so no need to generalise remaining terms.}} \\
\begin{align}
    p_{23}^{(n)} &= \frac{\frac{1}{30}n^5-\frac{1}{3}n^4+\frac{3}{2}n^3-\frac{8}{3}n^2+\frac{37}{15}n}{3^n} 
\end{align}
So, 
\begin{align}
    \lim_{n\to\infty} p_{23}^{(n)} &= \lim_{n\to\infty} \frac{\frac{1}{30}n^5-\frac{1}{3}n^4+\frac{3}{2}n^3-\frac{8}{3}n^2+\frac{37}{15}n}{3^n} 
\end{align}
By L'Hospital's Rule,
\begin{align}
    \lim_{n\to\infty} p_{23}^{(n)} &= \lim_{n\to\infty} \frac{\frac{1}{6}n^4-\frac{4}{3}n^3+\frac{9}{2}n^2-\frac{16}{3}n+\frac{37}{15}}{3^n\log_{10} 3} \\
    &= \lim_{n\to\infty} \frac{\frac{2}{3}n^3-4n^2+9n-\frac{16}{3}}{3^n(\log_{10} 3)^2} \\
    &= \lim_{n\to\infty} \frac{2n^2-8n+9}{3^n(\log_{10} 3)^3} \\
    &= \lim_{n\to\infty} \frac{4n-8}{3^n(\log_{10} 3)^4} \\
    &= \lim_{n\to\infty} \frac{4}{3^n(\log_{10} 3)^5} \\
    &= 0. \;\;\;\;\;\;\;\;\; (\therefore \text{As}\ n\to\infty, \frac{1}{3^n} \to 0. )
\end{align}
\end{flushleft}
\end{document}

