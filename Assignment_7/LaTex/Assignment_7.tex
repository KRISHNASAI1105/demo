\documentclass[journal,12pt,twocolumn]{IEEEtran}
\usepackage{setspace}
\usepackage{gensymb}
\singlespacing
\usepackage[cmex10]{amsmath}

\usepackage{relsize}
\usepackage{amsthm}
\usepackage{hyperref}
\hypersetup{
    colorlinks=true,
    linkcolor=blue,
    filecolor=magenta,      
    urlcolor=cyan,
}
\urlstyle{same}
\usepackage{placeins}
\usepackage{newtxtext}
\usepackage{mathrsfs}
\usepackage{txfonts}
\usepackage{stfloats}
\usepackage{bm}
\usepackage{cite}
\usepackage{cases}
\usepackage{subfig}
\usepackage{longtable}
\usepackage{multirow}
\usepackage{enumitem}
\usepackage{mathtools}
\usepackage{steinmetz}
\usepackage{tikz}
\usepackage{circuitikz}
\usepackage{verbatim}
\usepackage{tfrupee}
\usepackage[breaklinks=true]{hyperref}
\usepackage{booktabs}
\usepackage{graphicx}
\usepackage{tkz-euclide}
\usetikzlibrary{shapes,backgrounds}
\usepackage{verbatim}
\usetikzlibrary{calc,math}
\usepackage{listings}
    \usepackage{color}                                            %%
    \usepackage{array}                                            %%
    \usepackage{longtable}                                        %%
    \usepackage{calc}                                             %%
    \usepackage{multirow}                                         %%
    \usepackage{hhline}                                           %%
    \usepackage{ifthen}                                           %%
    \usepackage{lscape}     
\usepackage{multicol}
\usepackage{chngcntr}
\usepackage{mdframed}
\DeclareMathOperator*{\Res}{Res}

\renewcommand\thesection{\arabic{section}}
\renewcommand\thesubsection{\thesection.\arabic{subsection}}
\renewcommand\thesubsubsection{\thesubsection.\arabic{subsubsection}}

\renewcommand\thesectiondis{\arabic{section}}
\renewcommand\thesubsectiondis{\thesectiondis.\arabic{subsection}}
\renewcommand\thesubsubsectiondis{\thesubsectiondis.\arabic{subsubsection}}


\hyphenation{op-tical net-works semi-conduc-tor}
\def\inputGnumericTable{}                                 %%

\lstset{
%language=C,
frame=single, 
breaklines=true,
columns=fullflexible
}
\usepackage{chngcntr}
\counterwithin{figure}{section}

\title{AI1103 \\ Assignment 4}
\author{Nagubandi Krishna Sai \\ MS20BTECH11014}

\begin{document}
\newtheorem{theorem}{Theorem}[section]
\newtheorem{problem}{Problem}
\newtheorem{proposition}{Proposition}[section]
\newtheorem{lemma}{Lemma}[section]
\newtheorem{corollary}[theorem]{Corollary}
\newtheorem{example}{Example}[section]
\newtheorem{definition}[problem]{Definition}

\newcommand{\BEQA}{\begin{eqnarray}}
\newcommand{\EEQA}{\end{eqnarray}}
\newcommand{\define}{\stackrel{\triangle}{=}}
\bibliographystyle{IEEEtran}
\raggedbottom
\setlength{\parindent}{0pt}
\providecommand{\mbf}{\mathbf}
\providecommand{\pr}[1]{\ensuremath{\Pr\left(#1\right)}}
\providecommand{\qfunc}[1]{\ensuremath{Q\left(#1\right)}}
\providecommand{\sbrak}[1]{\ensuremath{{}\left[#1\right]}}
\providecommand{\lsbrak}[1]{\ensuremath{{}\left[#1\right.}}
\providecommand{\rsbrak}[1]{\ensuremath{{}\left.#1\right]}}
\providecommand{\brak}[1]{\ensuremath{\left(#1\right)}}
\providecommand{\lbrak}[1]{\ensuremath{\left(#1\right.}}
\providecommand{\rbrak}[1]{\ensuremath{\left.#1\right)}}
\providecommand{\cbrak}[1]{\ensuremath{\left\{#1\right\}}}
\providecommand{\lcbrak}[1]{\ensuremath{\left\{#1\right.}}
\providecommand{\rcbrak}[1]{\ensuremath{\left.#1\right\}}}
\theoremstyle{remark}
\newtheorem{rem}{Remark}
\newcommand{\sgn}{\mathop{\mathrm{sgn}}}
\providecommand{\abs}[1]{\vert#1\vert}
\providecommand{\res}[1]{\Res\displaylimits_{#1}} 
\providecommand{\norm}[1]{\lVert#1\rVert}
%\providecommand{\norm}[1]{\lVert#1\rVert}
\providecommand{\mtx}[1]{\mathbf{#1}}
\providecommand{\mean}[1]{E[ #1 ]}
\providecommand{\fourier}{\overset{\mathcal{F}}{ \rightleftharpoons}}
%\providecommand{\hilbert}{\overset{\mathcal{H}}{ \rightleftharpoons}}
\providecommand{\system}{\overset{\mathcal{H}}{ \longleftrightarrow}}
	%\newcommand{\solution}[2]{\textbf{Solution:}{#1}}
\newcommand{\solution}{\noindent \textbf{Solution: }}
\newcommand{\cosec}{\,\text{cosec}\,}
\providecommand{\dec}[2]{\ensuremath{\overset{#1}{\underset{#2}{\gtrless}}}}
\newcommand{\myvec}[1]{\ensuremath{\begin{pmatrix}#1\end{pmatrix}}}
\newcommand{\mydet}[1]{\ensuremath{\begin{vmatrix}#1\end{vmatrix}}}
\numberwithin{equation}{subsection}
\makeatletter
\@addtoreset{figure}{problem}
\makeatother
\let\StandardTheFigure\thefigure
\let\vec\mathbf
\renewcommand{\thefigure}{\theproblem}
\def\putbox#1#2#3{\makebox[0in][l]{\makebox[#1][l]{}\raisebox{\baselineskip}[0in][0in]{\raisebox{#2}[0in][0in]{#3}}}}
     \def\rightbox#1{\makebox[0in][r]{#1}}
     \def\centbox#1{\makebox[0in]{#1}}
     \def\topbox#1{\raisebox{-\baselineskip}[0in][0in]{#1}}
     \def\midbox#1{\raisebox{-0.5\baselineskip}[0in][0in]{#1}}
\vspace{3cm}
\title{AI1103 \\ Assignment 7}
\author{Nagubandi Krishna Sai \\ MS20BTECH11014}
\maketitle
\newpage
\bigskip
\renewcommand{\thefigure}{\theenumi}
\renewcommand{\thetable}{\theenumi}

Download LaTex file from below link :  

\begin{lstlisting}
https://github.com/KRISHNASAI1105/demo/blob/main/Assignment_7/LaTex/Assignment_7.tex
\end{lstlisting}

\subsection*{\boldsymbol{Problem\ number\ CSIR\ UGC\ NET\ 2014\ Q.106}}
\begin{flushleft} Consider a Markov chain with state space {1,2,....,100}. Suppose states 2i and 2j communicate with each other and states 2i-1 and 2j-1 communicate with each other for every i,j = 1,2,...,50. Further suppose that $p^{(2)}_{3,3}$ > 0,$p^{(3)}_{4,4}$ > 0 and $p^{(7)}_{2,5}$ > 0. Then 
\begin{enumerate}
\item The Markov chain is irreducible.
\item The Markov chain is aperiodic.
\item State 8 is recurrent.
\item State 9 is recurrent.
\end{enumerate}
\end{flushleft}

\subsection*{\boldsymbol{Solution}}
\begin{flushleft} \vspace{-7mm}
\begin{definition} We say that Markov chain is irreducible if and only if all states belong to one communication class and all states communicate with each other. 
\end{definition} 
\begin{definition}  In an \textbf{irreducible chain} all states belong to a single communicating class. This means that, if one of the states in an irreducible Markov chain is \textbf{aperiodic}. Then, all the remaining states are also aperiodic.
\end{definition} 
\begin{center}
S = \{1,2,....,100\}. 
\end{center}

\begin{tikzpicture}
\filldraw[color=red!60, fill=red!5, very thick](1.5,0) circle (0.4);
\filldraw[color=red!60, fill=red!5, very thick](0,1.5) circle (0.4);
\filldraw[color=red!60, fill=red!5, very thick](-1.5,0) circle (0.4);
\filldraw[color=red!60, fill=red!5, very thick](0,-1.5) circle (0.4);
\node[roundnode] at (1.5,0) {3};
\node[roundnode] at (0,1.5) {1};
\node[roundnode] at (-1.5,0) {49};
\node[roundnode] at (0,-1.5) {5};
\node (A) at (0, 1.5) {};
\node (B) at (1.5, 0) {};
\draw[->, to path={-\ (\tikztotarget)}]
  (A) edge (B) ;
\node (A) at (0, 1.5) {};
\node (B) at (-1.5, 0) {};
\draw[->, to path={-\ (\tikztotarget)}]
  (A) edge (B) ;
\node (A) at (0,-1.5) {};
\node (B) at (1.5, 0) {};
\draw[->, to path={-\ (\tikztotarget)}]
  (A) edge (B) ;
\node (A) at (0,-1.5) {};
\node (B) at (-1.5, 0) {};
\draw[->, to path={-\ (\tikztotarget)}]
  (A) edge (B) ;
\node (A) at (1.5,0) {};
\node (B) at (0,-1.5) {};
\draw[->, to path={-\ (\tikztotarget)}]
  (A) edge (B) ;
\node (A) at (1.5,0) {};
\node (B) at (0,1.5) {};
\draw[->, to path={-\ (\tikztotarget)}]
  (A) edge (B) ;
\node (A) at (-1.5,0) {};
\node (B) at (0,1.5) {};
\draw[->, to path={-\ (\tikztotarget)}]
  (A) edge (B) ;
\node (A) at (-1.5,0) {};
\node (B) at (0,-1.5) {};
\draw[->, to path={-\ (\tikztotarget)}]
  (A) edge (B) ;
\node (A) at (1.5,0) {};
\node (B) at (-1.5,0) {};
\draw[->, to path={-\ (\tikztotarget)}]
  (A) edge (B) ;
\node (A) at (-1.5,0) {};
\node (B) at (1.5,0) {};
\draw[->, to path={-\ (\tikztotarget)}]
  (A) edge (B) ;
\draw[green,ultra thick,dashed] (0,-1.7) -- (-1.8,0);


\filldraw[color=red!60, fill=red!5, very thick](3,0) circle (0.4);
\filldraw[color=red!60, fill=red!5, very thick](4.5,1.5) circle (0.4);
\filldraw[color=red!60, fill=red!5, very thick](6,0) circle (0.4);
\filldraw[color=red!60, fill=red!5, very thick](4.5,-1.5) circle (0.4);
\node[roundnode] at (3,0) {50};
\node[roundnode] at (4.5,1.5) {2};
\node[roundnode] at (6,0) {4};
\node[roundnode] at (4.5,-1.5) {6};
\node (A) at (4.5, 1.5) {};
\node (B) at (3, 0) {};
\draw[->, to path={-\ (\tikztotarget)}]
  (A) edge (B) ;
\node (A) at (4.5, 1.5) {};
\node (B) at (6, 0) {};
\draw[->, to path={-\ (\tikztotarget)}]
  (A) edge (B) ;
\node (A) at (4.5,-1.5) {};
\node (B) at (3, 0) {};
\draw[->, to path={-\ (\tikztotarget)}]
  (A) edge (B) ;
\node (A) at (4.5,-1.5) {};
\node (B) at (6, 0) {};
\draw[->, to path={-\ (\tikztotarget)}]
  (A) edge (B) ;
\node (A) at (3,0) {};
\node (B) at (4.5,-1.5) {};
\draw[->, to path={-\ (\tikztotarget)}]
  (A) edge (B) ;
\node (A) at (3,0) {};
\node (B) at (4.5,1.5) {};
\draw[->, to path={-\ (\tikztotarget)}]
  (A) edge (B) ;
\node (A) at (6,0) {};
\node (B) at (4.5,1.5) {};
\draw[->, to path={-\ (\tikztotarget)}]
  (A) edge (B) ;
\node (A) at (6,0) {};
\node (B) at (4.5,-1.5) {};
\draw[->, to path={-\ (\tikztotarget)}]
  (A) edge (B) ;
\node (A) at (3,0) {};
\node (B) at (6,0) {};
\draw[->, to path={-\ (\tikztotarget)}]
  (A) edge (B) ;
\node (A) at (-1.5,0) {};
\node (B) at (1.5,0) {};
\draw[->, to path={-\ (\tikztotarget)}]
  (A) edge (B) ;
\draw[green,ultra thick,dashed] (4.5,-1.6) -- (2.8,0);
\node (A) at (4.5,-1.7) {};
\node (B) at (1.5,0) {};
\draw[->, to path={-\ (\tikztotarget)}]
  (A) edge (B) ;

\filldraw[color=red!60, fill=red!5, very thick](0,-3) circle (0.4);
\filldraw[color=red!60, fill=red!5, very thick](1.5,-3) circle (0.4);
\filldraw[color=red!60, fill=red!5, very thick](3,-3) circle (0.4);
\filldraw[color=red!60, fill=red!5, very thick](6,-3) circle (0.4);
\node[roundnode] at (0,-3) {51};
\node[roundnode] at (1.5,-3) {52};
\node[roundnode] at (3,-3) {53};
\node[roundnode] at (6,-3) {100};
\draw[green,ultra thick,dashed] (3.5,-3.2) -- (5.5,-3.2);

\end{tikzpicture}

Consider, the communication classes of the given Markov chain as follows : \\
C_1(1) = \{ 1,3,5,7,....,49 \}. \\
C_1(2) = \{ 2,4,6,8,....,50 \}. \\
C_1(51) = \{ 51 \}, \ C_1(52) = \{ 52 \}, \ ...... \ C_1(100) = \{ 100 \}. \\
$\therefore$ As there are 52 communication classes, the given Markov chain is reducible. \\

\implies {[i \in C_{(i)}]} \ {($\because$ i communicate with i in zero steps)} \\

Given, even states communicate with each other. Similarly odd states communicate with each other. \\ 

Regarding periodicity, \\
\begin{center}
\begin{math}
d(K) = gcd(m \ge 1 : P^{m}_{k,k} > 0). 
\end{math}
\end{center} \\

For all odd states in \{1,3,5,7,....,49\}, Periodicity = d(1). \\
Similarly, For all even states in \{2,4,6,8,....,50\}, Periodicity = d(2). \\
d(1) = d(2) = gcd\{2,3,....\} = 1. $\therefore$ Aperiodic. \\
d(51) = d(52) = ....... = d(100) = 0. $\therefore$ periodic. \\

Hence, The given \textbf{Markov\ chain} is reducible and not a aperiodic chain. \\
\{1,3,5,7,...,49,51,52,53,....,100\} are recurrent states. \\
\{2,4,6,8,....,50\} are transient states.

\begin{center}
     \boxed{\text{\textbf{Option 4} is a \textbf{correct} answer}}
\end{center}

\end{flushleft}
\end{document}