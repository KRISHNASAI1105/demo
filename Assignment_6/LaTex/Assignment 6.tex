\documentclass[journal,12pt,twocolumn]{IEEEtran}
\usepackage{setspace}
\usepackage{gensymb}
\singlespacing
\usepackage[cmex10]{amsmath}

\usepackage{relsize}
\usepackage{amsthm}
\usepackage{hyperref}
\hypersetup{
    colorlinks=true,
    linkcolor=blue,
    filecolor=magenta,      
    urlcolor=cyan,
}
\urlstyle{same}
\usepackage{placeins}
\usepackage{newtxtext}
\usepackage{mathrsfs}
\usepackage{txfonts}
\usepackage{stfloats}
\usepackage{bm}
\usepackage{cite}
\usepackage{cases}
\usepackage{subfig}
\usepackage{longtable}
\usepackage{multirow}
\usepackage{enumitem}
\usepackage{mathtools}
\usepackage{steinmetz}
\usepackage{tikz}
\usepackage{circuitikz}
\usepackage{verbatim}
\usepackage{tfrupee}
\usepackage[breaklinks=true]{hyperref}
\usepackage{booktabs}
\usepackage{graphicx}
\usepackage{tkz-euclide}
\usetikzlibrary{shapes,backgrounds}
\usepackage{verbatim}
\usetikzlibrary{calc,math}
\usepackage{listings}
    \usepackage{color}                                            %%
    \usepackage{array}                                            %%
    \usepackage{longtable}                                        %%
    \usepackage{calc}                                             %%
    \usepackage{multirow}                                         %%
    \usepackage{hhline}                                           %%
    \usepackage{ifthen}                                           %%
    \usepackage{lscape}     
\usepackage{multicol}
\usepackage{chngcntr}
\usepackage{mdframed}
\DeclareMathOperator*{\Res}{Res}

\renewcommand\thesection{\arabic{section}}
\renewcommand\thesubsection{\thesection.\arabic{subsection}}
\renewcommand\thesubsubsection{\thesubsection.\arabic{subsubsection}}

\renewcommand\thesectiondis{\arabic{section}}
\renewcommand\thesubsectiondis{\thesectiondis.\arabic{subsection}}
\renewcommand\thesubsubsectiondis{\thesubsectiondis.\arabic{subsubsection}}


\hyphenation{op-tical net-works semi-conduc-tor}
\def\inputGnumericTable{}                                 %%

\lstset{
%language=C,
frame=single, 
breaklines=true,
columns=fullflexible
}
\usepackage{chngcntr}
\counterwithin{figure}{section}

\title{AI1103 \\ Assignment 4}
\author{Nagubandi Krishna Sai \\ MS20BTECH11014}

\begin{document}
\newtheorem{theorem}{Theorem}[section]
\newtheorem{problem}{Problem}
\newtheorem{proposition}{Proposition}[section]
\newtheorem{lemma}{Lemma}[section]
\newtheorem{corollary}[theorem]{Corollary}
\newtheorem{example}{Example}[section]
\newtheorem{definition}[problem]{Definition}

\newcommand{\BEQA}{\begin{eqnarray}}
\newcommand{\EEQA}{\end{eqnarray}}
\newcommand{\define}{\stackrel{\triangle}{=}}
\bibliographystyle{IEEEtran}
\raggedbottom
\setlength{\parindent}{0pt}
\providecommand{\mbf}{\mathbf}
\providecommand{\pr}[1]{\ensuremath{\Pr\left(#1\right)}}
\providecommand{\qfunc}[1]{\ensuremath{Q\left(#1\right)}}
\providecommand{\sbrak}[1]{\ensuremath{{}\left[#1\right]}}
\providecommand{\lsbrak}[1]{\ensuremath{{}\left[#1\right.}}
\providecommand{\rsbrak}[1]{\ensuremath{{}\left.#1\right]}}
\providecommand{\brak}[1]{\ensuremath{\left(#1\right)}}
\providecommand{\lbrak}[1]{\ensuremath{\left(#1\right.}}
\providecommand{\rbrak}[1]{\ensuremath{\left.#1\right)}}
\providecommand{\cbrak}[1]{\ensuremath{\left\{#1\right\}}}
\providecommand{\lcbrak}[1]{\ensuremath{\left\{#1\right.}}
\providecommand{\rcbrak}[1]{\ensuremath{\left.#1\right\}}}
\theoremstyle{remark}
\newtheorem{rem}{Remark}
\newcommand{\sgn}{\mathop{\mathrm{sgn}}}
\newcommand{\comb}[2]{{}^{#1}\mathrm{C}_{#2}}
\providecommand{\abs}[1]{\vert#1\vert}
\providecommand{\res}[1]{\Res\displaylimits_{#1}} 
\providecommand{\norm}[1]{\lVert#1\rVert}
%\providecommand{\norm}[1]{\lVert#1\rVert}
\providecommand{\mtx}[1]{\mathbf{#1}}
\providecommand{\mean}[1]{E[ #1 ]}
\providecommand{\fourier}{\overset{\mathcal{F}}{ \rightleftharpoons}}
%\providecommand{\hilbert}{\overset{\mathcal{H}}{ \rightleftharpoons}}
\providecommand{\system}{\overset{\mathcal{H}}{ \longleftrightarrow}}
	%\newcommand{\solution}[2]{\textbf{Solution:}{#1}}
\newcommand{\solution}{\noindent \textbf{Solution: }}
\newcommand{\cosec}{\,\text{cosec}\,}
\providecommand{\dec}[2]{\ensuremath{\overset{#1}{\underset{#2}{\gtrless}}}}
\newcommand{\myvec}[1]{\ensuremath{\begin{pmatrix}#1\end{pmatrix}}}
\newcommand{\mydet}[1]{\ensuremath{\begin{vmatrix}#1\end{vmatrix}}}
\numberwithin{equation}{subsection}
\makeatletter
\@addtoreset{figure}{problem}
\makeatother
\let\StandardTheFigure\thefigure
\let\vec\mathbf
\renewcommand{\thefigure}{\theproblem}
\def\putbox#1#2#3{\makebox[0in][l]{\makebox[#1][l]{}\raisebox{\baselineskip}[0in][0in]{\raisebox{#2}[0in][0in]{#3}}}}
     \def\rightbox#1{\makebox[0in][r]{#1}}
     \def\centbox#1{\makebox[0in]{#1}}
     \def\topbox#1{\raisebox{-\baselineskip}[0in][0in]{#1}}
     \def\midbox#1{\raisebox{-0.5\baselineskip}[0in][0in]{#1}}
\vspace{3cm}
\title{AI1103 \\ Assignment 6}
\author{Nagubandi Krishna Sai \\ MS20BTECH11014}
\maketitle
\newpage
\bigskip
\renewcommand{\thefigure}{\theenumi}
\renewcommand{\thetable}{\theenumi}

Download Python codes from below link :  
\begin{lstlisting}
https://github.com/KRISHNASAI1105/demo/blob/main/Assignment%205/code/Assignment%205.py
\end{lstlisting}
%
Download LaTex file from below link :  
%
\begin{lstlisting}
https://github.com/KRISHNASAI1105/demo/blob/main/Assignment%205/LaTex/Assignment%205.tex
\end{lstlisting}

\subsection*{\boldsymbol{Problem\ number\ CSIR\ UGC\ NET\ 2014\ Q.104}}
\begin{flushleft} Suppose $X_1$,$X_2$,$X_3$ and $X_4$ are independent and identically distributed random variables, having density function f. Then,
\begin{enumerate}
\item \pr{X_4 > Max(X_1,X_2) > X_3} = $\frac{1}{6}$
\item \pr{X_4 > Max(X_1,X_2) > X_3} = $\frac{1}{8}$
\item \pr{X_4 > X_3 > Max(X_1,X_2)} = $\frac{1}{12}$
\item \pr{X_4 > X_3 > Max(X_1,X_2)} = $\frac{1}{6}$
\end{enumerate}
\end{flushleft}

\subsection*{\boldsymbol{Solution}}
\begin{flushleft}
\begin{align*}
    Pr{(X2 > X1)} &= \int\limits_{-\infty}^\infty f_X(x) \int\limits_{-\infty}^x                  f_X(t)\ d t d x \\
                  &= \int\limits_{-\infty}^\infty f_X(x) F_X(x) d x \\
                  &= \cfrac{F^{2}_X(x)}{2} \biggr \vert_{-\infty}^{\infty} \\
                  &= \dfrac{1}{2} \\
\end{align*}
\begin{align*}
Pr{(X_4 > Max(X_1,X_2) > X_3)} &= \int\limits_{-\infty}^\infty f_X(x)                                             \int\limits_{-\infty}^x f _X(t).\comb{2}{1}. \\
                           &  {\Bigg[\int\limits_{-\infty}^t f_X(w) d w\Bigg]} \int\limits_{-\infty}^t f_X(z)\ d z d t d x \\       &= \int\limits_{-\infty}^\infty f_X(x) 
                              \int\limits_{-\infty}^x 2f_X(t)F^{2}_X(t)\ d t d x\\
                           &= \int\limits_{-\infty}^\infty f_X(x).\dfrac{2}{3}
                              F^{3}_X(x) d x \\
                           &= \dfrac{2}{3}\cfrac{F^{4}_X(x)}{4} \biggr \vert_
                              {-\infty}^{\infty} \\    
                           &= \dfrac{1}{6}   
\end{align*} 
\begin{align*}
Pr{(X_4 > X_3 > Max(X_1,X_2))} &= \int\limits_{-\infty}^\infty f_X(x) 
                                  \int\limits_{-\infty}^x f_X(t) 
                                  \int\limits_{-\infty}^t f_X(z). \\ 
                               &  \comb{2}{1}
                                  {\Bigg[\int\limits_{-\infty}^t f_X(w) d w\Bigg]}\ d z d t d x \\
                               &= \int\limits_{-\infty}^\infty f_X(x) 
                                  \int\limits_{-\infty}^x f_X(t) \\
                               &  \int\limits_{-\infty}^t 2f_X(z)F_X(t)d z d t d x \\
                               &= \int\limits_{-\infty}^\infty f_X(x) 
                                  \int\limits_{-\infty}^x f_X(t)F^{2}_X(t)\ d t d x \\
                               &= \int\limits_{-\infty}^\infty f_X(x).
                                  \dfrac{1}{3}F^{3}_X(x) d x \\
                               &= \dfrac{1}{3}\cfrac{F^{4}_X(x)}{4} \biggr \vert    _{-\infty}^{\infty} \\
                               &= \dfrac{1}{12}   
\end{align*}

\begin{center}
  $\therefore$ \boxed{\text{\textbf{Option 1,3} are \textbf{correct} answers.}}
\end{center}
\end{flushleft}
\end{document}