\documentclass[journal,12pt,twocolumn]{IEEEtran}
\usepackage{setspace}
\usepackage{gensymb}
\singlespacing
\usepackage[cmex10]{amsmath}

\usepackage{relsize}
\usepackage{amsthm}
\usepackage{hyperref}
\hypersetup{
    colorlinks=true,
    linkcolor=blue,
    filecolor=magenta,      
    urlcolor=cyan,
}
\urlstyle{same}
\usepackage{placeins}
\usepackage{newtxtext}
\usepackage{mathrsfs}
\usepackage{txfonts}
\usepackage{stfloats}
\usepackage{bm}
\usepackage{cite}
\usepackage{cases}
\usepackage{subfig}
\usepackage{longtable}
\usepackage{multirow}
\usepackage{enumitem}
\usepackage{mathtools}
\usepackage{steinmetz}
\usepackage{tikz}
\usepackage{circuitikz}
\usepackage{verbatim}
\usepackage{tfrupee}
\usepackage[breaklinks=true]{hyperref}
\usepackage{booktabs}
\usepackage{graphicx}
\usepackage{tkz-euclide}
\usetikzlibrary{shapes,backgrounds}
\usepackage{verbatim}
\usetikzlibrary{calc,math}
\usepackage{listings}
    \usepackage{color}                                            %%
    \usepackage{array}                                            %%
    \usepackage{longtable}                                        %%
    \usepackage{calc}                                             %%
    \usepackage{multirow}                                         %%
    \usepackage{hhline}                                           %%
    \usepackage{ifthen}                                           %%
    \usepackage{lscape}     
\usepackage{multicol}
\usepackage{chngcntr}
\usepackage{mdframed}
\DeclareMathOperator*{\Res}{Res}

\renewcommand\thesection{\arabic{section}}
\renewcommand\thesubsection{\thesection.\arabic{subsection}}
\renewcommand\thesubsubsection{\thesubsection.\arabic{subsubsection}}

\renewcommand\thesectiondis{\arabic{section}}
\renewcommand\thesubsectiondis{\thesectiondis.\arabic{subsection}}
\renewcommand\thesubsubsectiondis{\thesubsectiondis.\arabic{subsubsection}}


\hyphenation{op-tical net-works semi-conduc-tor}
\def\inputGnumericTable{}                                 %%

\lstset{
%language=C,
frame=single, 
breaklines=true,
columns=fullflexible
}
\usepackage{chngcntr}
\counterwithin{figure}{section}

\title{AI1103 \\ Assignment 4}
\author{Nagubandi Krishna Sai \\ MS20BTECH11014}

\begin{document}
\newtheorem{theorem}{Theorem}[section]
\newtheorem{problem}{Problem}
\newtheorem{proposition}{Proposition}[section]
\newtheorem{lemma}{Lemma}[section]
\newtheorem{corollary}[theorem]{Corollary}
\newtheorem{example}{Example}[section]
\newtheorem{definition}[problem]{Definition}

\newcommand{\BEQA}{\begin{eqnarray}}
\newcommand{\EEQA}{\end{eqnarray}}
\newcommand{\define}{\stackrel{\triangle}{=}}
\bibliographystyle{IEEEtran}
\raggedbottom
\setlength{\parindent}{0pt}
\providecommand{\mbf}{\mathbf}
\providecommand{\pr}[1]{\ensuremath{\Pr\left(#1\right)}}
\providecommand{\qfunc}[1]{\ensuremath{Q\left(#1\right)}}
\providecommand{\sbrak}[1]{\ensuremath{{}\left[#1\right]}}
\providecommand{\lsbrak}[1]{\ensuremath{{}\left[#1\right.}}
\providecommand{\rsbrak}[1]{\ensuremath{{}\left.#1\right]}}
\providecommand{\brak}[1]{\ensuremath{\left(#1\right)}}
\providecommand{\lbrak}[1]{\ensuremath{\left(#1\right.}}
\providecommand{\rbrak}[1]{\ensuremath{\left.#1\right)}}
\providecommand{\cbrak}[1]{\ensuremath{\left\{#1\right\}}}
\providecommand{\lcbrak}[1]{\ensuremath{\left\{#1\right.}}
\providecommand{\rcbrak}[1]{\ensuremath{\left.#1\right\}}}
\theoremstyle{remark}
\newtheorem{rem}{Remark}
\newcommand{\sgn}{\mathop{\mathrm{sgn}}}
\providecommand{\abs}[1]{\vert#1\vert}
\providecommand{\res}[1]{\Res\displaylimits_{#1}} 
\providecommand{\norm}[1]{\lVert#1\rVert}
%\providecommand{\norm}[1]{\lVert#1\rVert}
\providecommand{\mtx}[1]{\mathbf{#1}}
\providecommand{\mean}[1]{E[ #1 ]}
\providecommand{\fourier}{\overset{\mathcal{F}}{ \rightleftharpoons}}
%\providecommand{\hilbert}{\overset{\mathcal{H}}{ \rightleftharpoons}}
\providecommand{\system}{\overset{\mathcal{H}}{ \longleftrightarrow}}
	%\newcommand{\solution}[2]{\textbf{Solution:}{#1}}
\newcommand{\solution}{\noindent \textbf{Solution: }}
\newcommand{\cosec}{\,\text{cosec}\,}
\providecommand{\dec}[2]{\ensuremath{\overset{#1}{\underset{#2}{\gtrless}}}}
\newcommand{\myvec}[1]{\ensuremath{\begin{pmatrix}#1\end{pmatrix}}}
\newcommand{\mydet}[1]{\ensuremath{\begin{vmatrix}#1\end{vmatrix}}}
\numberwithin{equation}{subsection}
\makeatletter
\@addtoreset{figure}{problem}
\makeatother
\let\StandardTheFigure\thefigure
\let\vec\mathbf
\renewcommand{\thefigure}{\theproblem}
\def\putbox#1#2#3{\makebox[0in][l]{\makebox[#1][l]{}\raisebox{\baselineskip}[0in][0in]{\raisebox{#2}[0in][0in]{#3}}}}
     \def\rightbox#1{\makebox[0in][r]{#1}}
     \def\centbox#1{\makebox[0in]{#1}}
     \def\topbox#1{\raisebox{-\baselineskip}[0in][0in]{#1}}
     \def\midbox#1{\raisebox{-0.5\baselineskip}[0in][0in]{#1}}
\vspace{3cm}
\title{AI1103 \\ Assignment 5}
\author{Nagubandi Krishna Sai \\ MS20BTECH11014}
\maketitle
\newpage
\bigskip
\renewcommand{\thefigure}{\theenumi}
\renewcommand{\thetable}{\theenumi}

Download Python codes from below link :  
\begin{lstlisting}
https://github.com/KRISHNASAI1105/demo/blob/main/Assignment%205/code/Assignment%205.py
\end{lstlisting}
%
Download LaTex file from below link :  
%
\begin{lstlisting}
https://github.com/KRISHNASAI1105/demo/blob/main/Assignment%205/LaTex/Assignment%205.tex
\end{lstlisting}

\subsection*{\boldsymbol{Problem\ number\ GATE\ EC\ 2019\ Q.20}}
\begin{flushleft} Let Z be an exponential random variable with mean 1. That is, the cumulative distribution function of Z is given by 
\begin{align*}
    F_Z(x) =
    \begin{cases}
    1 - {e^{-x}}, & if\ x \ge 0\\
    0, & if\ x<0.
    \end{cases}
\end{align*}

Then Pr{(Z>2 | Z>1)}, rounded off to two decimal places, is equal to 
\end{flushleft}

\subsection*{\boldsymbol{Solution}}
\begin{flushleft}
Given that Z is an exponential distribution with cumulative function F_Z(x).

\begin{figure}[!htb]
\centering
\includegraphics[width = 0.44\textwidth]{CDF of X.png}
\end{figure}

We know that probability density function 
\begin{align*}
  f_Z(x) = F^\prime _Z(x) = \begin{cases}{e^{-x}},& x \ge 0 \\
                          0, & x<0
  \end{cases}
\end{align*}

The CDF of $X$ is,
\begin{align*}
             F_Z(x) &= Pr(Z \le x),\ for\ all\ x \in R \\
        Pr(Z \le 2) &= F_Z(2) \\
                    &= 1 - {e^{-2}} \\
        Pr(Z \le 1) &= F_Z(1) \\
                    &= 1 - {e^{-1}} \\
          Pr(Z > 2) &= 1 - Pr(Z \le 2) \\
                    &= {e^{-2}} \\
          Pr(Z > 1) &= 1 - Pr(Z \le 1) \\
                    &= {e^{-1}} \\        
    Pr{(Z>2 | Z>1)} &= \dfrac{Pr{((Z>2)\cap (Z>1))}}{Pr{(Z>1)}} \\
                    &= \dfrac{Pr{(Z>2)}}{Pr{(Z>1)}} \\
                    &= \dfrac{e^{-2}}{e^{-1}} \\
                    &= e^{-1} \\
                    &= 0.3679
\end{align*}


\begin{figure}[!htb]
\centering
\includegraphics[width = 0.44\textwidth]{Probability.png}
\end{figure}

\end{flushleft}
\end{document}
