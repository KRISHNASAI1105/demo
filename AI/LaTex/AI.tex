\documentclass{report}
\usepackage[utf8]{inputenc}
\usepackage{tikz}
\usepackage{mathtools}  
\usepackage{diffcoeff}
\usepackage{pgfplots}
\usepackage{comment}
\usepackage{graphicx}
\usepackage{blindtext}
\usepackage{subcaption}
\usepackage[export]{adjustbox}
\usepackage{wrapfig}
\usepackage{rotating}

\title{\textbf{Machine Learning}}
\author{\textbf{Nagubandi Krishna Sai} \\ \textbf{MS20BTECH11014}}
\date{\textbf{June 2021}}

\graphicspath{{images/}}

\begin{document}

\maketitle

% \section{Contents}

\tableofcontents{}

\chapter{Fundamentals of Machine Learning}
\section{Supervised Learning}
Supervised Learning gives "correct answers", the output values are same as real life values. \\
In Supervised Learning, we are given a set of data and we know what our correct output should look like, having an idea that there is relationship between the input and the output. \\
Supervised Learning problems has two types of problems, 
\begin{enumerate}
    \item Regression. 
    \item Classification. 
\end{enumerate} \\

\subsection{Linear Regression}
In regression type of problems, we are trying to predict results within a continuous output, means that we are trying to map input variables to some continuous function. \\
Linear regression has real-valued output, but the output will be same or near valued to the actual output. 
\begin{enumerate}
    \item m = Number of training examples. 
    \item x's = "input" variable (or) feature.
    \item y's = "output (or) target" variable.
    \item (x,y) = one training example.
    \item ($x^{(i)}$,$y^{(i)}$) = $i^{th}$ training example.
\end{enumerate} 
Example : \newline
\begin{table}[h!]
    \centering
    \begin{tabular}{|c|c|} \hline
        Size of feet^2 (x) & Price(\$) in 1000's (y)  \\ \hline
        2104 & 460  \\ \hline
        1416 & 232  \\ \hline
        1534 & 315  \\ \hline
        852 & 178  \\ \hline
        \vdots & \vdots  \\ \hline
    \end{tabular}
    \caption{Training set of housing prices.}
    \label{tab:my_label}
\end{table}
\begin{enumerate}
    \item m = 47.
    \item $x^{(1)}$ = 2104 and $y^{(1)}$ = 460. 
    \item $x^{(2)}$ = 1416 and $y^{(2)}$ = 232.
\end{enumerate}
\begin{center}
\begin{tikzpicture}
\node[draw] at (2.5,2.5) {Training set};
\node[draw] at (2.5,0.5) {Learning algorithm};
\node[draw] at (2.5,-2.5) {h};
\node (A) at (2.5,2.3) {};
\node (B) at (2.5,0.7) {};
\draw[->, to path={-\ (\tikztotarget)}]
  (A) edge (B) ;
\node (A) at (2.5,0.3) {};
\node (B) at (2.5,-2.3) {};
\draw[->, to path={-\ (\tikztotarget)}]
  (A) edge (B) ;
\node[draw] at (6.5,1.5) {Training set is feeded to learning algorithm};
\node[draw] at (9,-2.5) {(hypothesis, This is a function that takes input x and estimate value of y.)}
\end{tikzpicture}
\end{center} \\
\subsection{Hypothesis for Linear Regression}
\begin{equation}
    h_\theta(x) = \theta_0 + \theta_1.x 
\end{equation} 
\begin{center}
    $\theta_i$'s = parameters.
\end{center} \\
This type of hypothesis mode is called "Linear regression with one variable" (or) "Univariate linear regression."  \\
\subsection{Cost function for Linear Regression}
Cost function helps us know that how well to fit the best possible straight line over the given data. \\ \newline
Q\rangle\; How\ to\ choose\ $\theta_i$'s\ ? 
\begin{enumerate}
    \item Choose\ $\theta_i$ 's so that $h_\theta(x)$ is close to y for our training example (x,y). 
    \item minimise $\theta_0$,$\theta_1$ so, that [$h_\theta(x)$ - y] is small.
\end{enumerate}
\begin{equation}
    J(\theta_0,\theta_1) = \frac{1}{2m}\sum_{i=1}^{m}[h_\theta(x^{(i)}) - y^{(i)}]^2
\end{equation}
\begin{center}
    J($\theta_0$,$\theta_1$) = Cost function (or) Squared error function.
\end{center} \\
\subsection{Gradient descent for Linear Regression}
Gradient descent is used to minimise cost function(J) in linear regression. \\
Gradient descent is used in many areas to minimise many functions in ML/AI. \\
\textbf{Gradient descent algorithm,} \\
\begin{equation}
    \text{Repeat\ until\ convergence\ (minimum)} \Bigg\{ \theta_j := \theta_j - \alpha\frac{\partial}{\partial \theta_j} J(\theta_0,\theta_1),\ for\ j=0, j=1.
\end{equation}
\begin{enumerate}
    \item := is Assignment operator.
    \item $\alpha$ is learning rate.
    \item $\frac{\partial}{\partial \theta_j} J(\theta_0,\theta_1)$ is derivative. 
\end{enumerate} \\
Gradient descent is nothing but the derivative of the Cost function. \\
\begin{equation}
    Slope\ of\ cost\ function\ curve = \frac{\partial J(\theta_1)}{\partial \theta_1},\ when\ \theta_0 = 0.
\end{equation} \\
\textbf{Learning rate,} \\
\begin{enumerate}
    \item If $\alpha$ is too small, gradient descent can be slow. After many such operations(can be infinite times), the '$\theta_1$' could reach "global minimum".
    \item If $\alpha$ is too large, gradient descent can overshoot the minimum. It may "fail to converge (or) even diverge". 
    \item If $\theta_1$ is at the local optima itself when we started or taken $\theta_1$, then there is no use of "$\alpha$ (or) gradient descent". 
\end{enumerate}
\subsection{Linear Regression for multivariables}
\textbf{Hypothesis,} \\
\begin{equation}
    h_\theta(x) = \theta_0 + \theta_1 x_1 + \theta_2 x_2 + ... + \theta_n x_n.
\end{equation} \\
In the total context of Supervised learning, hypothesis is just predicting the output. \\
For convenience of notation, declare x_0 = 1\ (x^{(i)}_0 = 1). \\
\begin{math} 
x = \left[\begin{array}{c}
     x_0 \\
     x_1 \\
     x_2 \\
     \vdots \\
     x_n
\end{array}\right] \epsilon\ \Re^{n+1}  \;\;\;\;\;\;\;\;\;  
\theta = \left[\begin{array}{c}
     \theta_0 \\
     \theta_1 \\
     \theta_2 \\
     \vdots \\
     \theta_n
\end{array}\right] \epsilon\ \Re^{n+1}
\end{math} \\
The above matrix is 0 - indexed. \\
\begin{equation}
    h_\theta(x) = \theta_0 + \theta_1 x_1 + \theta_2 x_2 + ... + \theta_n x_n. \\
    h_\theta(x) = \theta^{\top} x.
\end{equation} \\
\textbf{Cost function,} \\
\begin{equation}
    J(\theta) = J(\theta_0,\theta_1,\theta_2,...,\theta_n) = \frac{1}{2m}\sum_{i=1}^{m}[h_\theta(x^{(i)}) - y^{(i)}]^2
\end{equation} \\
\textbf{Gradient descent,} 
\begin{center}
\begin{align}
    \theta_j &:= \theta_j - \alpha\frac{\partial}{\partial \theta_j} J(\theta_0,\theta_1,\theta_2,...,\theta_n)\\
    \theta_j &:= \theta_j - \alpha\frac{\partial}{\partial \theta_j} J(\theta) \\
    \theta_j &:= \theta_j - \alpha\frac{1}{m}\sum_{i=1}^{m}[h_\theta(x^{(i)}) - y^{(i)}]x^{(i)}_j
\end{align} 
\end{center} \\
\textbf{Feature scaling,} \\
Get every feature into approximately -1\le x_i \le 1 \ range. \\
\textbf{Mean normalization,} \\
Replace x_i \ with\ x_i - \mu_i\ to\ make\ features\ have\ approximately\ zero\ mean. \\
\subsection{Polynomial regression}
\textbf{Hypothesis,} \\
\begin{equation}
    h_\theta(x) = \theta_0 + \theta_1x_1 + \theta_2 x_2 + \theta_3 x_3.
\end{equation}\\
\begin{table}[h!]
    \centering
    \begin{tabular}{|c|} \hline
         Housing price prediction  \\ \hline 
         x_1 = size \\ \hline
         x_2 = (size)^2 \\ \hline
         x_3 = (size)^3 \\  \hline
    \end{tabular}
    \caption{Features be-like in Polynomial regression.}
    \label{tab:my_label}
\end{table}
\subsection{Normal Equation}
\textbf{Intuition,} \\
\begin{center}
    \begin{math}
    \diff{}{\theta}J(\theta) = 0. 
    \end{math}
\end{center} \\
\textbf{Cost function,} 
\begin{center} \vspace{-9mm}
    \begin{align} 
    \theta\ \epsilon\ \Re^{n+1},\ J(\theta_0,\theta_1,\theta_2,...,\theta_n) = \frac{1}{2m} \sum_{i=1}^{m} [h_\theta(x^{(i)}) - y{(i)}]^2 \\
    \frac{\partial }{\partial \theta_j} J(\theta) = 0,\ (solve\ for\ \theta_0\ ,\theta_1\ ,...\ ,\theta_n)
    \end{align} \\
\end{center}
\textbf{Example,} \\
\begin{table}[h!]
    \centering
    \begin{tabular}{|c|c|c|c|c|c|} \hline
         & Size (feet^2) & No.of Bed rooms & No.of floors & Age of home (years) & Price(\$1000) \\ \hline
        x_0 & x_1 & x_2 & x_3 & x_4 & y \\ \hline 
        1 & 2104 & 5 & 1 & 45 & 460 \\ \hline
        1 & 1416 & 3 & 2 & 40 & 232 \\ \hline
        1 & 1534 & 3 & 2 & 30 & 315 \\ \hline
        1 & 1852 & 2 & 1 & 36 & 178 \\ \hline
    \end{tabular}
    \caption{Sample Training set for Multi-Variate Linear regression.}
    \label{tab:my_label}
\end{table} 
\begin{enumerate}
    \item n = number of Features. 
    \item $x_j^{(i)}$ = value of j in the $i^{th}$ training example.
    \item $x_{(i)}$ = the input(features) of the $i^{th}$ training example.
\end{enumerate} \\
\begin{math}
X = \left[\begin{array}{ccccc}
    1 & 2104 & 5 & 1 & 45 \\
    1 & 1416 & 3 & 2 & 40 \\
    1 & 1534 & 3 & 2 & 30 \\
    1 & 852 & 2 & 1 & 36
\end{array} \right] &\;\;\;\;\;\;\;\;\; Y = \left[\begin{array}{c}
    460  \\
    232 \\
    315 \\
    178 
\end{array} \right]
\end{math} \\
X is m$\times$(n+1)-dimensional\ matrix\ and\ Y\ is\ a\ m-dimensional\ vector. \\
\begin{align}
\theta &= {(X^{\top}X)}^{-1} X^{\top} Y
\end{align} \\
The above $\theta$\ value\ is\ optimal\ $\theta$\ value. \\
For Normal equation method, then no need to use \textbf{feature scaling.} \\
We use 'Gradient descent' and 'Normal equation' methods to minimise cost function. \\
\begin{table}[h!]
    \centering
    \begin{tabular}{|c|c|} \hline
        Gradient descent & Normal equation  \\ \hline
        1$\rangle$ Need to choose '$\alpha$'. & 1$\rangle$ No need to choose '$\alpha$'. \\ \hline
        2$\rangle$ Need many iterations. & 2$\rangle$ Don't need to iterate. \\ \hline
        3$\rangle$ Works well even, when 'n' is large. (n>10000) & 3$\rangle$ Need to compute n$\times$ n matrix inverse {(X^{$\top$}X)}^{-1} \\ \hline
         & 4$\rangle$ Works Now if n is very large. \\ \hline
    \end{tabular}
    \caption{Why should we use the particular method? Advantages and Disadvantages of two methods.}
    \label{tab:my_label}
\end{table} 
\begin{align}
\theta &= {(X^{\top}X)}^{-1} X^{\top} Y
\end{align} \\
Q$\rangle$ What is $X^{\top}X$ is non-invertible(singular/degenerate) ? \\
Reasons, 
\begin{enumerate}
    \item Redundant features (linearly dependent)
    \begin{itemize}
        \item $x_1$ = Size in $feet^2$ 
        \item $x_2$ = Size in $m^2$
        \item $x_2$ = ${(3.28)}^2$ $x_1$ , 1m = 3.28feet. 
    \end{itemize}
    \item Too many features. (m$\le$n)
    \begin{itemize}
        \item m = 10
        \item n = 100, $\theta\ \epsilon\ {\Re}^{101}$, Delete some features (or) Regularization.
    \end{itemize}
\end{enumerate} \\
\subsection{Classification} 
The output value 'y' is \textbf{discrete value.} \\
The algorithm used is \textbf{logistic regression.} \\
\subsection{Logistic Regression Model}
$\therefore$ We want $0\le h_\theta(x) \le 1$. \\
\begin{align}
    h_\theta(x) &= g(\theta^{\top} x) \\
    g(z) &= \frac{1}{1+e^{-z}} \\
    h_\theta(x) &= g(\theta^{\top} x) \\
                &= \frac{1}{1+e^{-\theta^{\top} x}}
\end{align} \\
The above g(z) is called sigmoid function (or) logistic function. \\
\textbf{Graph,} \\
\begin{center}
\begin{tikzpicture}
\begin{axis}
\addplot[color=red]{1/(1+exp(-x))};
\end{axis}
\end{tikzpicture}
\end{center}
\begin{align}
    g(z) &\ge 0.5,\ when\ z \ge 0. \\
    h_\theta(x) = g(\theta^{\top} x) &\ge 0.5,\ when\ \theta^{\top} x \ge 0.
\end{align} 
\subsection{Interpretation of hypothesis output for Logistic Regression}
\begin{align}
    h_\theta(x) &= P(y=1|x;\theta) \\
    P(y=0|x;\theta) &+ P(y=1|x;\theta) = 1 \\
    P(y=0|x;\theta) &= 1-P(y=1|x;\theta) \\
                    &= 1-h_\theta(x) \\
\end{align} 
\therefore\;\;\;\;\;\;\;\;\;\;\;\;\;\;\;\;\;\;\;\;\;\;\;\;\;\;\;\;\;\;\;\;\;\;\;\;\;\;\;\;\;\;\;\;\;\;\;\;\;\; y = 0 (or) 1. \\
\textbf{Decision boundary,} \\
\begin{center}
\begin{tikzpicture}
\begin{axis}[
    enlargelimits=false,
]
\addplot[
    only marks,
    scatter,
    mark=x,
    mark size=2.9pt]
table[meta=ma]
{scattered_example.dat};
\end{axis}
\end{tikzpicture}
\end{center} \\
\begin{figure}[!htb]
\centering
\includegraphics[width = 0.45\textwidth]{Screenshot (79).png}
\end{figure} \\
For, the above diagram decision boundary will be a line separating the two output values of y(y=0 (or) y=1). \\
\begin{align}
    h_\theta(x) &\ge 0.5 \rightarrow y=1. \\
    h_\theta(x) &< 0.5 \rightarrow y=0. \\
    h_\theta(x) &= g(\theta^{\top} x) \\
                &= g(\theta_0 + \theta_1x_1 + \theta_2x_2)
\end{align} 
\begin{math}
\theta = \left[\begin{array}{c}
    \theta_0  \\
    \theta_1 \\
    \theta_2
\end{array} \right] & \;\;\;\;\;\;\;\;\;\;\;\;\;\;\;\;\;\;\;\;\;\;\;\;\;\;\;\;\; x = \left[\begin{array}{c}
    1  \\
    x_1 \\
    x_2
\end{array} \right]
\end{math} \\
\textbf{Example,} \\
Let, \\
\begin{center}
\begin{math}
\theta = \left[\begin{array}{c}
    -3  \\
    1 \\
    1
\end{array} \right] 
\end{math}
\end{center}
\begin{align}
    y = 1, if\ \theta^{\top} x &\ge 0 \\
               -3+x_1+x_2 &\ge 0 \\
               x_1+x_2 &\ge 3,\ y=1 \\
               x_1+x_2 &<3,\ y=0.
\end{align} 
\textbf{Non-Linear Decision Boundary,} \\
\begin{figure}[!htb]
\centering
\includegraphics[width = 1.2\textwidth]{Screenshot (83).png}
\end{figure} \\
So, in the above non-linear classification, the \textbf{decision boundary is a circle} of radius of 1unit. \\
\begin{align}
    Inside\ circle,\ y &= 0 \\
    Outside\ circle,\ y &= 1.
\end{align} \\
\subsection{Cost function for Logistic Regression}
\textbf{Training set,} \\
\begin{table}[h!]
    \centering
    \begin{tabular}{|c|c|} \hline
        x^{(1)} & y^{(1)} \\ \hline
        x^{(2)} & y^{(2)} \\ \hline
        x^{(3)} & y^{(3)} \\ \hline
        \vdots & \vdots \\ \hline
        x^{(m)} & y^{(m)} \\ \hline
    \end{tabular}
    \caption{Training set of m-examples for Logistic Regression}
    \label{tab:my_label}
\end{table} \\
\begin{math}
x = \left[\begin{array}{c}
     x_0 \\
     x_1 \\
     x_2 \\
     \vdots \\
     x_n
\end{array}\right] \epsilon\ \Re^{n+1}\ -\ n\ features.\ x_0 = 1,\ y\ \epsilon\ {0,1}. 
\end{math} \\
For linear regression, \\
\begin{align}
    J(\theta) = \frac{1}{2m} \sum_{i=1}^{m} [h_\theta(x^{(i)} - y^{(i)}]^2 \\
    J(\theta) = \frac{1}{m} \sum_{i=1}^{m} Cost(h_\theta(x^{(i)}, y^{(i)}) \\
    Cost(h_\theta(x^{(i)}), y^{(i)}) = \frac{1}{2} [h_\theta(x^{(i)} - y^{(i)}]^2
\end{align} \\
\begin{center}
\begin{tikzpicture}
\begin{axis}[title={Convex function},
    xlabel={x-axis},
    ylabel={y-axis}]
\addplot[color=red]{x^2};
\end{axis}
\end{tikzpicture}
\end{center}
The above graph is a convex. \\
\begin{figure}[!htb]
\centering
\includegraphics[width = 0.45\textwidth]{Non-convex(10.png}
\end{figure} \\
The below graph is a non-convex.\\
If we use the cost function of linear regression in logistic regression, the the we would get non-convex cost function, because the \textbf{hypothesis is} $\frac{1}{1+e^{-\theta^{\top}x}}$. \\
For Logistic regression, 
\begin{align}
    Cost(h_\theta(x^{(i)}), y^{(i)}) =
    \begin{cases}
    -\log(1-h_\theta(x)), & if\ y = 0\\
    -\log(h_\theta(x)), & if\ y=1.
    \end{cases}
\end{align} \\
If y=1, \\
% \begin{center}
% \begin{tikzpicture}
% \begin{axis}[title={Convex function},
%     xlabel={h_\theta(x)}
%     ylabel={Cost function}]
% \addplot[color=red]{-ln(x)};
% \end{axis}
% \end{tikzpicture}
% \end{center}
\begin{center}
\setlength{\unitlength}{0.8cm}
\begin{picture}(6,6)
\thicklines
\put(1,1){\line(1,0){5}}
\put(1,1){\line(0,1){5}}
\qbezier(1.3,6)(1.5,1.693147)(6,1)
\put(3,0.5){$h_\theta(x)$}
\put(0.5,2.5){\begin{turn}{90}Cost function\end{turn}}
\put(5.95,0.9){$|$}
\put(5.95,0.3){1}
\put(0.95,0.9){$|$}
\put(1,0.3){0}
\end{picture}
\end{center} \\
If y=0, \\
\begin{center}
\setlength{\unitlength}{0.8cm}
\begin{picture}(6,6)
\thicklines
\put(1,1){\line(1,0){5}}
\put(1,1){\line(0,1){5}}
\qbezier(1,1)(5,1.3)(6,6)
\put(3,0.5){$h_\theta(x)$}
\put(0.5,2.5){\begin{turn}{90}Cost function\end{turn}}
\put(5.95,0.9){$|$}
\put(5.95,0.3){1}
\put(0.95,0.9){$|$}
\put(1,0.3){0}
\end{picture}
\end{center} \\
\textbf{Simplified Cost function,} \\
\begin{align}
    Cost(h_\theta(x^{(i)}), y^{(i)}) &= -(1-y)\log(1-h_\theta(x)) -y\log(h_\theta(x)).\ \forall\ y\ \epsilon\ \{0,1\}. \\
    If\ y &= 1: Cost(h_\theta(x), y) = -\log(h_\theta(x)). \\
    If\ y &= 0: Cost(h_\theta(x), y) = -\log(1-h_\theta(x)). 
\end{align}
\begin{align}
    Cost function = J(\theta) &= \frac{1}{m} \sum_{i=1}^{m} Cost(h_\theta(x^{(i)}, y^{(i)}) \\
    &= \frac{1}{m} [-\sum_{i=1}^{m} (1-y^{(i)})\log(1-h_\theta(x^{(i)})) +y^{(i)} \log(h_\theta(x^{(i)}))] 
\end{align}
\subsection{Gradient Descent for Logistic Regression}
\begin{align}
    \text{Repeat\ until\ convergence\ (minimum)}& \Bigg\{ \theta_j := \theta_j - \alpha\frac{\partial}{\partial \theta_j} J(\theta) \\
    \frac{\partial}{\partial \theta_j} J(\theta) &= \frac{1}{m} \sum_{i=1}^{m} [h_\theta(x^{(i)}) - y^{(i)}]x_j^{(i)}
\end{align}
\subsection{Optimization algorithm}
\begin{enumerate}
    \item Gradient descent. 
    \item Conjugate gradient.
    \item BFGS.
    \item L - BFGS.
\end{enumerate}
These are the 4 algorithms to minimise \textbf{cost function.} \\
Advantages of the \textbf{last three advanced optimization algorithm.} 
\begin{itemize}
    \item No need to manually pick $\alpha$.
    \item Often faster than Gradient descent.
    \item They themselves choose $\alpha$, for faster convergence.
\end{itemize} 
\subsection{Multiclass Classification : One-vs-All}
\begin{align*}
y &\in \lbrace0, 1 ... n\rbrace \\
h_\theta^{(0)}(x) &= P(y = 0 | x ; \theta) \\
h_\theta^{(1)}(x) &= P(y = 1 | x ; \theta) \\
&\vdots \\
h_\theta^{(n)}(x) &= P(y = n | x ; \theta) \\
\mathrm{prediction} &= \max_i( h_\theta ^{(i)}(x) ) 
\end{align*}
\begin{figure}[!htb]
\centering
\includegraphics[width = 1.2\textwidth]{multi class.png}
\end{figure} 
To summarize, 
\begin{enumerate}
    \item Train a logistic regression classifier $h_\theta(x)$ for each class to predict the probability that y=i.
    \item To make a prediction on a new x, pick the class that maximizes  $h_\theta(x)$
\end{enumerate}
\subsection{Problem of Overfitting,} 
\begin{figure}[!htb]
\centering
\includegraphics[width = 1.2\textwidth]{Screenshot (89).png}
\end{figure} \\
Similar for Logistic Regression. \\
\textbf{Addressing Overfitting,} 
\begin{enumerate}
    \item Reduce number of features.
    \begin{itemize}
        \item Manually select which features to keep.
        \item Model selection algorithm.
    \end{itemize}
    \item Regularization
    \begin{itemize}
        \item Keep all features, but reduce magnitude/values of parameters $\thicklines_j$.
        \item Works well when we have a lot of features, each of which contributes a bit to predict $y^{(i)}$.
    \end{itemize}
\end{enumerate}
\subsection{Regularization}
Small values for parameters $\theta_0,\theta_1,\theta_2,...,\theta_n$.
\begin{enumerate}
    \item Simpler hypothesis.
    \item Less prone to overfitting.
\end{enumerate}
\textbf{Regularized Cost function,}
\begin{align}
    J(\theta) = \frac{1}{2m} \sum_{i=1}^{m} \bigg[[h_\theta(x^{(i)} - y^{(i)}]^2 + \lambda\sum_{j=1}^{n} \theta_j^2\bigg] 
\end{align} \\
If '$\lambda$' is extremely large, then the cost function will become underfitting (doesn't fit to our training data). \\
\begin{align*} 
\text{Repeat}\ \lbrace & \\
\theta_0 &:= \theta_0 - \alpha\ \frac{1}{m}\ \sum_{i=1}^m (h_\theta(x^{(i)}) - y^{(i)})x_0^{(i)} \\
\theta_j &:= \theta_j - \alpha\ \left[ \left( \frac{1}{m}\ \sum_{i=1}^m (h_\theta(x^{(i)}) - y^{(i)})x_j^{(i)} \right) + \frac{\lambda}{m}\theta_j \right] &\ \ \ \ \ \ \ \ \ \ j \in \lbrace 1,2...n\rbrace\\
\rbrace
\end{align*}
The term $\frac{\lambda}{m} \theta_j$ performs our regularization. With some manipulation our update rule can also be represented as:
\begin{align}
\theta_j := \theta_j(1 - \alpha\frac{\lambda}{m}) - \alpha\frac{1}{m}\sum_{i=1}^m(h_\theta(x^{(i)}) - y^{(i)})x_j^{(i)}
\end{align}
The first term in the above equation, $1 - \alpha\frac{\lambda}{m}$ will always be less than 1. Intuitively you can see it as reducing the value of $\theta_j$ by some amount on every update. Notice that the second term is now exactly the same as it was before. \\
\textbf{Normal equation after regularization,}
\begin{align}
\theta &= {(X^{\top}X) + \lambda.L}^{-1} X^{\top} Y
\end{align} 
\begin{math}
where\ L = \left[\begin{array}{ccccc}
    0 &  &  &  &   \\
     & 1 &  &  &   \\
     &  & 1 &  & \\
     &  &  & \ddots &  \\
     &  &  &  & 1 \\
\end{array} \right]
\end{math} 
\subsection{Regularized Logistic Regression}
\begin{figure}[!htb]
\centering
\includegraphics[width = 1.2\textwidth]{regular logis 1.png}
\end{figure}
\textbf{Cost function,} 
\begin{align}
J(\theta) = \frac{1}{m} \bigg[-\sum_{i=1}^{m} (1-y^{(i)})\log(1-h_\theta(x^{(i)})) +y^{(i)} \log(h_\theta(x^{(i)}))\bigg] + \frac{\lambda}{2m} \sum_{j=1}^{n} \theta_j^2
\end{align} 
\newpage The second sum, $\sum_{j=1}^n \theta_j^2$ means to explicitly exclude the bias term, $\theta_0$. I.e. the θ vector is indexed from 0 to n (holding n+1 values, $\theta_0$ through $\theta_n$), and this sum explicitly skips $\theta_0$, by running from 1 to n, skipping 0. Thus, when computing the equation, we should continuously update the two following equations: 
\begin{align*} 
\text{Repeat}\ \lbrace & \\
\theta_0 &:= \theta_0 - \alpha\ \frac{1}{m}\ \sum_{i=1}^m (h_\theta(x^{(i)}) - y^{(i)})x_0^{(i)} \\
\theta_j &:= \theta_j - \alpha\ \left[ \left( \frac{1}{m}\ \sum_{i=1}^m (h_\theta(x^{(i)}) - y^{(i)})x_j^{(i)} \right) + \frac{\lambda}{m}\theta_j \right] &\ \ \ \ \ \ \ \ \ \ j \in \lbrace 1,2...n\rbrace\\
\rbrace
\end{align*}
\section{Neural Networks}
\subsection{Model representation I}
Let's examine how we will represent a hypothesis function using neural networks. At a very simple level, neurons are basically computational units that take inputs (dendrites) as electrical inputs (called \textbf{"spikes"}) that are channeled to outputs (axons). In our model, our dendrites are like the input features $x_1\cdots x_n$, and the output is the result of our hypothesis function. In this model our $x_0$ input node is sometimes called the \textbf{"bias unit".} It is always equal to 1. In neural networks, we use the same logistic function as in classification, $\frac{1}{1 + e^{-\theta^Tx}}$, yet we sometimes call it a sigmoid (logistic) activation function. In this situation, our \textbf{"theta" parameters} are sometimes called \textbf{"weights".} \\
A simple representation looks like : 
\begin{center}
\begin{math}
\left[ \begin{array}{c}
    x_0x_1x_2  \\
\end{array}\right] \rightarrow \left[\; \right] \rightarrow h_\theta(x)
\end{math}
\end{center} \\
Our input nodes (layer 1), also known as the \textbf{"input layer"}, go into another node (layer 2), which finally outputs the hypothesis function, known as the \textbf{"output layer".} \\
We can have intermediate layers of nodes between the input and output layers called the \textbf{"hidden layers."} \\
In this example, we label these intermediate or \textbf{"hidden"} layer nodes $a^2_0, \cdots, a^2_na$ and call them \textbf{"activation units."} 
\begin{center}
    \begin{enumerate}
        \item $a_i^{(j)}$ = "activation" of unit i in layer j
        \item $\Theta^{(j)}$ = matrix of weights controlling function mapping from layer j to layer j+1
    \end{enumerate}
\end{center}
If we had one hidden layer, it would look like : \\
\begin{center}
\begin{math}
\left[ \begin{array}{c}
    x_0x_1x_2x_3  \\
\end{array}\right] \rightarrow \left[a_1^{(2)}a_2^{(2)}a_3^{(2)} \right] \rightarrow h_\theta(x)
\end{math}
\end{center}
The values for each of the \textbf{"activation"} nodes is obtained as follows : \\
\begin{align}
    a_1^{(2)} = g(\Theta_{10}^{(1)}x_0 + \Theta_{11}^{(1)}x_1 + \Theta_{12}^{(1)}x_2 + \Theta_{13}^{(1)}x_3) \\
    a_2^{(2)} = g(\Theta_{20}^{(1)}x_0 + \Theta_{21}^{(1)}x_1 + \Theta_{22}^{(1)}x_2 + \Theta_{23}^{(1)}x_3) \\
    a_3^{(2)} = g(\Theta_{30}^{(1)}x_0 + \Theta_{31}^{(1)}x_1 + \Theta_{32}^{(1)}x_2 + \Theta_{33}^{(1)}x_3) \\
    h_\Theta(x) = a_1^{(3)} = g(\Theta_{10}^{(2)}a_0^{(2)} + \Theta_{11}^{(2)}a_1^{(2)} + \Theta_{12}^{(2)}a_2^{(2)} + \Theta_{13}^{(2)}a_3^{(2)})
\end{align}
This is saying that we compute our activation nodes by using a 3$\times$4 matrix of parameters. We apply each row of the parameters to our inputs to obtain the value for one activation node. Our hypothesis output is the logistic function applied to the sum of the values of our activation nodes, which have been multiplied by yet another parameter matrix $\Theta^{(2)}$ containing the weights for our second layer of nodes. \\
Each layer gets its own matrix of weights, $\Theta^{(j)}$. \\
The dimensions of these matrices of weights is determined as follows : \\ 
If network has $s_j$ units in layer $j$ and $s_{j+1}$ units in layer $j+1$, then $\Theta^{(j)}$ will be of dimension $s_{j+1} \times (s_j + 1)$. \\
The +1 comes from the addition in $\Theta^{(j)}$ of the \textbf{"bias nodes,"} $x_0$ and $\Theta_0^{(j)}$. In other words the output nodes will not include the bias nodes while the inputs will. \\ 
\subsection{Model representation II}
we'll do a vectorized implementation of the above functions. We're going to define a new variable $z_k^{(j)}$ that encompasses the parameters inside our g function. \\
\begin{align}
    a_1^{(2)} &= g(z_1^{(2)}) \\
    a_2^{(2)} &= g(z_2^{(2)}) \\
    a_3^{(2)} &= g(z_3^{(2)})
\end{align}
In other words, for layer j=2 and node k, the variable z will be : \\
\begin{align}
    z_k^{(2)} = \Theta_{k,0}^{(1)} x_0 + \Theta_{k,1}^{(1)} x_1 + \cdots + \Theta_{k,n}^{(1)} x_n
\end{align}
The vector representation of x and $z^{j}$ is : \\
\begin{math}
x = \left[\begin{array}{c}
x_0 \\
x_1 \\
\vdots \\
x_n
\end{array} \right] \;\;\;\;\;\;\;\;\;\;\;\;\;\;\;\;\;\;\;\;\;\;\;\;\;\;\;\;\;
z^{(j)} = \left[\begin{array}{c}
z_1^{(j)} \\
z_2^{(j)} \\
\vdots \\
z_n^{(j)}
\end{array} \right]
\end{math} \\
Setting x = $a^{(1)}$, we can rewrite the equation as :
\begin{align}
    z^{(j)} = \theta^{(j-1)} a^{(j-1)}
\end{align} 
We are multiplying our matrix $\Theta^{(j-1)}$ with dimensions $s_j\times$ (n+1)(where $s_j$ is the number of our activation nodes) by our vector $a^{(j-1)}$ with height (n+1). This gives us our vector $z^{(j)}$ with height $s_j$. Now we can get a vector of our activation nodes for layer j as follows : \\
\begin{align} 
a^{(j)} = g(z^{(j)})a 
\end{align}
Where our function g can be applied element-wise to our vector $z^{(j)}$ \\
We can then add a bias unit (equal to 1) to layer j after we have computed $a^{(j)}$. This will be element $a_0^{(j)}$ and will be equal to 1. To compute our final hypothesis, let's first compute another z vector : \\
\begin{align}
z^{(j+1)} = \Theta^{(j)}a^{(j)}
\end{align}
We get this final z vector by multiplying the next theta matrix after $\Theta^{(j-1)}$ with the values of all the activation nodes we just got. This last theta matrix $\Theta^{(j)}$ will have only one row which is multiplied by one column $a^{(j)}$ so that our result is a single number. We then get our final result with : \\
\begin{align}
h_\Theta(x) = a^{(j+1)} = g(z^{(j+1)})
\end{align}
Notice that in this last step, between layer j and layer j+1, we are doing exactly the same thing as we did in logistic regression. Adding all these intermediate layers in neural networks allows us to more elegantly produce interesting and more complex non-linear hypothesis. \\
\textbf{Examples,} \\
A simple example of applying neural networks is by predicting $x_1$ AND $x_2$, which is the logical 'and' operator and is only true if both $x_1$ and $x_2$ are 1.
\begin{align*}
h_\Theta(x) &= g(-30 + 20x_1 + 20x_2) \\
x_1 &= 0 \ \ and \ \ x_2 = 0 \ \ then \ \ g(-30) \approx 0 \\
x_1 &= 0 \ \ and \ \ x_2 = 1 \ \ then \ \ g(-10) \approx 0 \\
x_1 &= 1 \ \ and \ \ x_2 = 0 \ \ then \ \ g(-10) \approx 0 \\
x_1 &= 1 \ \ and \ \ x_2 = 1 \ \ then \ \ g(10) \approx 1
\end{align*}
So we have constructed one of the fundamental operations in computers by using a small neural network rather than using an actual AND gate. Neural networks can also be used to simulate all the other logical gates. The following is an example of the logical operator 'OR', meaning either $x_1$ is true or $x_2$ is true, or both :
\begin{figure}[!htb]
\centering
\includegraphics[width = 1.2\textwidth]{OR 1.png}
\end{figure} \\
The $Θ^{(1)}$ matrices for AND, NOR, and OR are : 
\begin{align*}
AND:\\
\Theta^{(1)} &=\begin{bmatrix}-30 & 20 & 20\end{bmatrix} \\
NOR:\\
\Theta^{(1)} &= \begin{bmatrix}10 & -20 & -20\end{bmatrix} \\
OR:\\
\Theta^{(1)} &= \begin{bmatrix}-10 & 20 & 20\end{bmatrix} \\
\end{align*}
We can combine these to get the XNOR logical operator (which gives 1 if $x_1$ and $x_2$ are both 0 or both 1). 
\begin{align*}
\begin{bmatrix}
x_0 \\
x_1 \\
x_2\end{bmatrix} 
\rightarrow \begin{bmatrix}
a_1^{(2)} \\
a_2^{(2)} \end{bmatrix}
\rightarrow\begin{bmatrix}
a^{(3)}\end{bmatrix} 
\rightarrow h_\Theta(x)
\end{align*}
Let's write out the values for all our nodes :
\begin{align*}
a^{(2)} = g(\Theta^{(1)} \cdot x) \\
a^{(3)} = g(\Theta^{(2)} \cdot a^{(2)}) \\
h_\Theta(x) = a^{(3)}
\end{align*}
\begin{figure}[!htb]
\centering
\includegraphics[width = 1.2\textwidth]{XNOR.png}
\end{figure} 
\subsection{Multiclass Classification}
To classify data into multiple classes, we let our hypothesis function return a vector of values. Say we wanted to classify our data into one of four categories. We will use the following example to see how this classification is done. This algorithm takes as input an image and classifies it accordingly : \\
\begin{figure}[!htb]
\centering
\includegraphics[width = 1.2\textwidth]{neural multi.png}
\end{figure} \\ \newpage
We can define our set of resulting classes as y : \\
\begin{figure}[h]
\centering
\includegraphics[width = 1.2\textwidth]{y.png}
\end{figure} \\
Each $y^{(i)}$ represents a different image corresponding to either a car, pedestrian, truck, or motorcycle. The inner layers, each provide us with some new information which leads to our final hypothesis function. The setup looks like : \\
\begin{figure}[!htb]
\centering
\includegraphics[width = 1.2\textwidth]{h.png}
\end{figure} \\ \newpage
Our resulting hypothesis for one set of inputs may look like : 
\begin{center}
    \begin{math}
    h_\theta(x) = \left[\begin{array}{c}
        0\ 0\ 1\ 0  \\
    \end{array} \right]
    \end{math}
\end{center} \\
In which case our resulting class is the third one down, or $h_\Theta(x)_3$, which represents the motorcycle. \\
\subsection{Cost function for Neural networks}
Let's declare some variables. 
\begin{enumerate}
    \item L = total number of layers in the network.
    \item $s_l$ = number of units (not counting bias unit) in layer l.
    \item K = number of output units/classes.
\end{enumerate}
\textbf{The cost function for regularized logistic regression,}
\begin{align}
    J(\theta) = \frac{1}{m} \bigg[-\sum_{i=1}^{m} (1-y^{(i)})\log(1-h_\theta(x^{(i)})) +y^{(i)} \log(h_\theta(x^{(i)}))\bigg] + \frac{\lambda}{2m} \sum_{j=1}^{n} \theta_j^2
\end{align} \\
\textbf{The cost function for Neural networks,}
\begin{align}
    J(\theta) = \frac{1}{2m} \sum_{i=1}^{m}\sum_{k=1}^{K} \bigg[(1-y^{(i)}_k)\log(1-(h_\theta(x^{(i)}))_k) +y^{(i)}_k \log((h_\theta(x^{(i)}))_k)\bigg] + \frac{\lambda}{2m} \sum_{l=1}^{L-1}\sum_{i=1}^{s_l}\sum_{j=1}^{s_{l+1}} (\Theta_{j,i}^{(l)})^2
\end{align} \\
In the regularization part, after the square brackets, we must account for multiple theta matrices. The number of columns in our current theta matrix is equal to the number of nodes in our current layer (including the bias unit). The number of rows in our current theta matrix is equal to the number of nodes in the next layer (excluding the bias unit). As before with logistic regression, we square every term. \\
\begin{enumerate}
    \item The double sum simply adds up the logistic regression costs calculated for each cell in the output layer.
    \item The triple sum simply adds up the squares of all the individual Θs in the entire network.
    \item The i in the triple sum does not refer to training example i.
\end{enumerate}
\subsection{Backpropagation Algorithm}
\textbf{"Backpropagation"} is neural-network terminology for minimizing our cost function, just like what we were doing with gradient descent in logistic and linear regression. Our goal is to compute : 
\begin{align}
    \min_\Theta J(\Theta)
\end{align}
That is, we want to minimize our cost function J using an optimal set of parameters in theta. In this section we'll look at the equations we use to compute the partial derivative of J(Θ) :
\begin{align}
    \frac{\partial}{\partial \Theta_{j,i}^{(l)}} J(\Theta)
\end{align}
\textbf{Backpropagation algorithm,}
\begin{figure}[!htb]
\centering
\includegraphics[width = 1.1\textwidth]{Back propagation.png}
\end{figure} 
\begin{enumerate}
    \item Given training set $\lbrace (x^{(1)}, y^{(1)}) \cdots (x^{(m)}, y^{(m)})\rbrace$.
    \begin{itemize}
        \item Set $\Delta^{(l)}_{i,j} :=$ 0 for all (l,i,j), (hence you end up having a matrix full of zeros)
    \end{itemize}
    \item For training example t = 1 to m :
    \begin{itemize}
        \item Set $a^{(1)} := x^{(t)}$.
        \item Perform forward propagation to compute $a^{(l)}$ for l=2,3,…,L.
        \item Using $y^{(t)}$, compute $\delta^{(L)} = a^{(L)} - y^{(t)}$.
        Where L is our total number of layers and $a^{(L)}$ is the vector of outputs of the activation units for the last layer. So our \textbf{"error values"} for the last layer are simply the differences of our actual results in the last layer and the correct outputs in y. To get the delta values of the layers before the last layer, we can use an equation that steps us back from right to left :
        \begin{figure}[!htb]
        \centering
        \includegraphics[width = 1.1\textwidth]{Back gradient.png}
        \end{figure} 
        \item Compute $\delta^{(L-1)}, \delta^{(L-2)},\dots,\delta^{(2)}$ using $\delta^{(l)} = ((\Theta^{(l)})^T \delta^{(l+1)})\ .*\ a^{(l)}\ .*\ (1 - a^{(l)})$. \\
        The delta values of layer l are calculated by multiplying the delta values in the next layer with the theta matrix of layer l. We then element-wise multiply that with a function called $g^\prime$, or g-prime, which is the derivative of the activation function g evaluated with the input values given by $z^{(l)}$. \\
        The g-prime derivative terms can also be written out as :
        \begin{align}
            g^\prime(z^{(l)}) = a^{(l)}\ .*\ (1 - a^{(l)})
        \end{align}
        \item  $\Delta^{(l)}_{i,j} := \Delta^{(l)}_{i,j} + a_j^{(l)} \delta_i^{(l+1)}$ (or) with vectorization, $\Delta^{(l)} := \Delta^{(l)} + \delta^{(l+1)}(a^{(l)})^T$. \\
        Hence we update our new $\Delta$ matrix.
        \begin{enumerate}
            \item $D_{i,j}^{(l)} := \frac{1}{m} (\Delta_{i,j}^{(l)} + \lambda\Theta_{i,j}^{(l)})$, if j\neq 0. \\
            \item $D_{i,j}^{(l)} := \frac{1}{m} (\Delta_{i,j}^{(l)}$, if j = 0.
        \end{enumerate}
        The capital-delta matrix D is used as an "accumulator" to add up our values as we go along and eventually compute our partial derivative. Thus we get $\frac{\partial}{\partial \Theta_{ij}^{(l)}} J(\Theta) = D_{ij}^{(l)}$.
    \end{itemize}
\end{enumerate}
If we consider simple non-multiclass classification (k = 1) and disregard regularization, the cost is computed with :
\begin{align}
    Cost(t) = (1-y^{(t)})\log(1-h_\theta(x^{(t)})) +y^{(t)} \log(h_\theta(x^{(t)}))
\end{align}
Intuitively, $\delta_j^{(l)}$ is the \textbf{"error"} for $a^{(l)}_j$ (unit j in layer l). More formally, the delta values are actually the derivative of the cost function :
\begin{align}
    \delta_{j}^{(l)} = \frac{\partial}{\partial z_j^{(l)}} cost(t)
\end{align}
Our derivative is the slope of a line tangent to the cost function, so the steeper the slope the more incorrect we are. Let us consider the following neural network below and see how we could calculate some $\delta_j^{(l)}$ :
\begin{figure}[!htb]
\centering
\includegraphics[width = 1.1\textwidth]{Forward.png}
\end{figure} \newpage
In the image above, to calculate $\delta_2^{(2)}$, we multiply the weights $\Theta_{12}^{(2)}$ and $\Theta_{22}^{(2)}$ by their respective $\delta$ values found to the right of each edge. So we get $\delta_2^{(2)} = \Theta_{12}^{(2)}*\delta_1^{(3)}+\Theta_{22}^{(2)}*\delta_2^{(3)}$. To calculate every single possible $\delta_j^{(l)}$, we could start from the right of our diagram. We can think of our edges as our $\Theta_{ij}$. Going from right to left, to calculate the value of $\delta_j^{(l)}$, you can just take the over all sum of each weight times the $\delta$ it is coming from. Hence, another example would be $\delta_2^{(3)} =\Theta_{12}^{(3)}*\delta_1^{(4)}$.
\subsection{Gradient Checking}
Gradient checking will assure that our backpropagation works as intended. We can approximate the derivative of our cost function with :
\begin{align}
    \frac{\partial}{\partial \theta} J(\Theta) \approx \frac{J(\Theta + \epsilon) - J(\Theta - \epsilon)}{2\epsilon}
\end{align}
With multiple theta matrices, we can approximate the derivative with respect to $\Theta_j$ as follows :
\begin{align}
    \frac{\partial}{\partial \theta_j} J(\Theta) \approx \frac{J(\Theta_1,...,\Theta_j + \epsilon,..., \Theta_n) - J(\Theta_1,...,\Theta_j - \epsilon,..., \Theta_n)}{2\epsilon}
\end{align}
A small value for ${\epsilon}$ (epsilon) such as ${\epsilon = 10^{-4}}$, guarantees that the math works out properly. If the value for $\epsilon$ is too small, we can end up with numerical problems. \\
We previously saw how to calculate the deltaVector. So once we compute our gradApprox vector, we can check that gradApprox $\approx$ deltaVector. \\
Once you have verified once that your backpropagation algorithm is correct, you don't need to compute gradApprox again. The code to compute gradApprox can be very slow.
\subsection{Random Intialization}
Initializing all theta weights to zero does not work with neural networks. When we backpropagate, all nodes will update to the same value repeatedly. Instead we can randomly initialize our weights for our $\Theta$ matrices using the following method :
\begin{figure}[!htb]
\centering
\includegraphics[width = 1.1\textwidth]{random.png}
\end{figure} 
\newpage Hence, we initialize each $\Theta^{(l)}_{ij}$ to a random value between $[-\epsilon,\epsilon]$. Using the above formula guarantees that we get the desired bound. \\
\boxed{\textbf{The epsilon used above is unrelated to the epsilon from Gradient Checking.}}
\subsection{Choosing Neural network}
\begin{enumerate}
    \item Pick a network architecture.
    \item Choose the layout of your neural network.
    \item Including how many hidden units in each layer and how many layers in total you want to have.
    \begin{itemize}
        \item Number of input units = dimension of features $x^{(i)}$.
        \item Number of output units = number of classes.
        \item Number of hidden units per layer = usually more the better (must balance with cost of computation as it increases with more hidden units).
        \item \textbf{Defaults} : 1 hidden layer. If you have more than 1 hidden layer, then it is recommended that you have the same number of units in every hidden layer.
    \end{itemize}
\end{enumerate}
\subsection{Training a Neural Network}
\begin{enumerate}
    \item Randomly initialize the weights.
    \item Implement forward propagation to get $h_\Theta(x^{(i)})$. \\
    \item Implement the cost function.
    \item Implement backpropagation to compute partial derivatives.
    \item Use gradient checking to confirm that your backpropagation works. Then disable gradient checking.
    \item Use gradient descent or a built-in optimization function to minimize the cost function with the weights in theta.
\end{enumerate}
\boxed{\textbf{Ideally, you want $h_\Theta(x^{(i)}) \approx≈ y^{(i)}$. This will minimize our cost function. However,}}
\boxed{\textbf{ keep in mind that $J(\Theta)$ is not convex and thus we can end up in a local minimum instead. }}
\end{document}