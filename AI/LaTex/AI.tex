\documentclass{report}
\usepackage[utf8]{inputenc}
\usepackage{tikz}
\usepackage{mathtools}  
\usepackage{diffcoeff}

\title{\textbf{Machine Learning}}
\author{\textbf{Nagubandi Krishna Sai} \\ \textbf{MS20BTECH11014}}
\date{\textbf{June 2021}}

\begin{document}

\maketitle

% \section{Contents}

\tableofcontents{}

\chapter{Fundamentals of Machine Learning}
\section{Supervised Learning}
Supervised Learning gives "correct answers", the output values are same as real life values. \\
In Supervised Learning, we are given a set of data and we know what our correct output should look like, having an idea that there is relationship between the input and the output. \\
Supervised Learning problems has two types of problems, 
\begin{enumerate}
    \item Regression. 
    \item Classification. 
\end{enumerate} \\

\subsection{Linear Regression}
In regression type of problems, we are trying to predict results within a continuous output, means that we are trying to map input variables to some continuous function. \\
Linear regression has real-valued output, but the output will be same or near valued to the actual output. \\
\begin{enumerate}
    \item m = Number of training examples. 
    \item x's = "input" variable (or) feature.
    \item y's = "output (or) target" variable.
    \item (x,y) = one training example.
    \item ($x^{(i)}$,$y^{(i)}$) = $i^{th}$ training example.
\end{enumerate} \\
Example : \\
\begin{table}[h!]
    \centering
    \begin{tabular}{|c|c|} \hline
        Size of feet^2 (x) & Price(\$) in 1000's (y)  \\ \hline
        2104 & 460  \\ \hline
        1416 & 232  \\ \hline
        1534 & 315  \\ \hline
        852 & 178  \\ \hline
        \vdots & \vdots  \\ \hline
    \end{tabular}
    \caption{Training set of housing prices.}
    \label{tab:my_label}
\end{table}
\begin{enumerate}
    \item m = 47.
    \item $x^{(1)}$ = 2104 and $y^{(1)}$ = 460. 
    \item $x^{(2)}$ = 1416 and $y^{(2)}$ = 232.
\end{enumerate}
\begin{center}
\begin{tikzpicture}
\node[draw] at (2.5,2.5) {Training set};
\node[draw] at (2.5,0.5) {Learning algorithm};
\node[draw] at (2.5,-2.5) {h};
\node (A) at (2.5,2.3) {};
\node (B) at (2.5,0.7) {};
\draw[->, to path={-\ (\tikztotarget)}]
  (A) edge (B) ;
\node (A) at (2.5,0.3) {};
\node (B) at (2.5,-2.3) {};
\draw[->, to path={-\ (\tikztotarget)}]
  (A) edge (B) ;
\node[draw] at (6.5,1.5) {Training set is feeded to learning algorithm};
\node[draw] at (9,-2.5) {(hypothesis, This is a function that takes input x and estimate value of y.)}
\end{tikzpicture}
\end{center} \\
\subsection{Hypothesis}
\begin{equation}
    h_\theta(x) = \theta_0 + \theta_1.x 
\end{equation} \\
\begin{center}
    $\theta_i$'s = parameters.
\end{center} \\
This type of hypothesis mode is called "Linear regression with one variable" (or) "Univariate linear regression."  \\
\subsection{Cost function}
Cost function helps us know that how well to fit the best possible straight line over the given data. \\ \newline
Q\rangle\; How\ to\ choose\ $\theta_i$'s\ ? 
\begin{enumerate}
    \item Choose\ $\theta_i$ 's so that $h_\theta(x)$ is close to y for our training example (x,y). 
    \item minimise $\theta_0$,$\theta_1$ so, that [$h_\theta(x)$ - y] is small.
\end{enumerate}
\begin{equation}
    J(\theta_0,\theta_1) = \frac{1}{2m}\sum_{i=1}^{m}[h_\theta(x^{(i)}) - y^{(i)}]^2
\end{equation}
\begin{center}
    J($\theta_0$,$\theta_1$) = Cost function (or) Squared error function.
\end{center} \\
\subsection{Gradient descent}
Gradient descent is used to minimise cost function(J) in linear regression. \\
Gradient descent is used in many areas to minimise many functions in ML/AI. \\
\textbf{Gradient descent algorithm,} \\
\begin{equation}
    \text{Repeat\ until\ convergence\ (minimum)} \Bigg\{ \theta_j := \theta_j - \alpha\frac{\partial}{\partial \theta_j} J(\theta_0,\theta_1),\ for\ j=0, j=1.
\end{equation}
\begin{enumerate}
    \item := is Assignment operator.
    \item $\alpha$ is learning rate.
    \item $\frac{\partial}{\partial \theta_j} J(\theta_0,\theta_1)$ is derivative. 
\end{enumerate} \\
Gradient descent is nothing but the derivative of the Cost function. \\
\begin{equation}
    Slope\ of\ cost\ function\ curve = \frac{\partial J(\theta_1)}{\partial \theta_1},\ when\ \theta_0 = 0.
\end{equation} \\
\textbf{Learning rate,} \\
\begin{enumerate}
    \item If $\alpha$ is too small, gradient descent can be slow. After many such operations(can be infinite times), the '$\theta_1$' could reach "global minimum".
    \item If $\alpha$ is too large, gradient descent can overshoot the minimum. It may "fail to converge (or) even diverge". 
    \item If $\theta_1$ is at the local optima itself when we started or taken $\theta_1$, then there is no use of "$\alpha$ (or) gradient descent". 
\end{enumerate}
\subsection{Linear Regression for multivariables}
\textbf{Hypothesis,} \\
\begin{equation}
    h_\theta(x) = \theta_0 + \theta_1 x_1 + \theta_2 x_2 + ... + \theta_n x_n.
\end{equation} \\
In the total context of Supervised learning, hypothesis is just predicting the output. \\
For convenience of notation, declare x_0 = 1\ (x^{(i)}_0 = 1). \\
\begin{math} 
x = \left[\begin{array}{c}
     x_0 \\
     x_1 \\
     x_2 \\
     \vdots \\
     x_n
\end{array}\right] \epsilon\ \Re^{n+1}  \;\;\;\;\;\;\;\;\;  
\theta = \left[\begin{array}{c}
     \theta_0 \\
     \theta_1 \\
     \theta_2 \\
     \vdots \\
     \theta_n
\end{array}\right] \epsilon\ \Re^{n+1}
\end{math} \\
The above matrix is 0 - indexed. \\
\begin{equation}
    h_\theta(x) = \theta_0 + \theta_1 x_1 + \theta_2 x_2 + ... + \theta_n x_n. \\
    h_\theta(x) = \theta^{\top} x.
\end{equation} \\
\textbf{Cost function,} \\
\begin{equation}
    J(\theta) = J(\theta_0,\theta_1,\theta_2,...,\theta_n) = \frac{1}{2m}\sum_{i=1}^{m}[h_\theta(x^{(i)}) - y^{(i)}]^2
\end{equation} \\
\textbf{Gradient descent,} \\
\begin{center}
\begin{align}
    \theta_j &:= \theta_j - \alpha\frac{\partial}{\partial \theta_j} J(\theta_0,\theta_1,\theta_2,...,\theta_n)\\
    \theta_j &:= \theta_j - \alpha\frac{\partial}{\partial \theta_j} J(\theta) \\
    \theta_j &:= \theta_j - \alpha\frac{1}{m}\sum_{i=1}^{m}[h_\theta(x^{(i)}) - y^{(i)}]x^{(i)}_j
\end{align} 
\end{center} \\
\textbf{Feature scaling,} \\
Get every feature into approximately -1\le x_i \le 1 \ range. \\
\textbf{Mean normalization,} \\
Replace x_i \ with\ x_i - \mu_i\ to\ make\ features\ have\ approximately\ zero\ mean. \\
\subsection{Polynomial regression}
\textbf{Hypothesis,} \\
\begin{equation}
    h_\theta(x) = \theta_0 + \theta_1x_1 + \theta_2 x_2 + \theta_3 x_3.
\end{equation}\\
\begin{table}[h!]
    \centering
    \begin{tabular}{|c|} \hline
         Housing price prediction  \\ \hline 
         x_1 = size \\ \hline
         x_2 = (size)^2 \\ \hline
         x_3 = (size)^3 \\  \hline
    \end{tabular}
    \caption{Features be-like in Polynomial regression.}
    \label{tab:my_label}
\end{table}
\subsection{Normal Equation}
\textbf{Intuition,} \\
\begin{center}
    \begin{math}
    \diff{}{\theta}J(\theta) = 0. 
    \end{math}
\end{center} \\
\textbf{Cost function,} 
\begin{center} \vspace{-9mm}
    \begin{align} 
    \theta\ \epsilon\ \Re^{n+1},\ J(\theta_0,\theta_1,\theta_2,...,\theta_n) = \frac{1}{2m} \sum_{i=1}^{m} [h_\theta(x^{(i)}) - y{(i)}]^2 \\
    \frac{\partial }{\partial \theta_j} J(\theta) = 0,\ (solve\ for\ \theta_0\ ,\theta_1\ ,...\ ,\theta_n)
    \end{align} \\
\end{center}
\textbf{Example,} \\
\begin{table}[h!]
    \centering
    \begin{tabular}{|c|c|c|c|c|c|} \hline
         & Size (feet^2) & No.of Bed rooms & No.of floors & Age of home (years) & Price(\$1000) \\ \hline
        x_0 & x_1 & x_2 & x_3 & x_4 & y \\ \hline 
        1 & 2104 & 5 & 1 & 45 & 460 \\ \hline
        1 & 1416 & 3 & 2 & 40 & 232 \\ \hline
        1 & 1534 & 3 & 2 & 30 & 315 \\ \hline
        1 & 1852 & 2 & 1 & 36 & 178 \\ \hline
    \end{tabular}
    \caption{Sample Training set for Multi-Variate Linear regression.}
    \label{tab:my_label}
\end{table} \\
\begin{enumerate}
    \item n = number of Features. 
    \item $x_j^{(i)}$ = value of j in the $i^{th}$ training example.
    \item $x_{(i)}$ = the input(features) of the $i^{th}$ training example.
\end{enumerate} \\
\begin{math}
X = \left[\begin{array}{ccccc}
    1 & 2104 & 5 & 1 & 45 \\
    1 & 1416 & 3 & 2 & 40 \\
    1 & 1534 & 3 & 2 & 30 \\
    1 & 852 & 2 & 1 & 36
\end{array} \right] &\;\;\;\;\;\;\;\;\; Y = \left[\begin{array}{c}
    460  \\
    232 \\
    315 \\
    178 
\end{array} \right]
\end{math} \\
X is m\times(n+1)-dimensional\ matrix\ and\ Y\ is\ a\ m-dimensional\ vector. \\
\begin{center}
\begin{equation} \vspace{-10mm}
    \theta = {(X^{\top}X)}^{-1} X^{\top} Y
\end{equation} 
\end{center}\\
\end{document}